\documentclass{article}
%\documentclass{ctexart}
\usepackage{multicol}
\usepackage{wrapfig}
\usepackage{lipsum}
%for long table
\makeatletter
\newenvironment{tablehere}
  {\def\@captype{table}}
 {}

\newenvironment{figurehere}
 {\def\@captype{figure}}
 {}
\makeatother
\usepackage{longtable}
%\documentclass{ctexart}
\usepackage{ctex}%中文
\usepackage{graphicx}
\usepackage{zhlipsum}
\usepackage{color}
\usepackage{multirow}
\usepackage{biblatex}%biber才可以
\addbibresource{ass.bib}
\usepackage{cuted}
\graphicspath{{picture/}}
\usepackage[left=2.5cm,right=1.97cm,top=2.5cm,bottom=2.5cm]{geometry}%设置页边距
\renewcommand{\baselinestretch}{1.25}%行间距
%\usepackage[hidelinks,urlcolor=black,linkcolor=black]{hyperref}%引入超链接包,否则会出现Undefined sequence
\usepackage[hidelinks]{hyperref}
%[colorlinks,urlcolor=black,linkcolor=black]去除超链接中的颜色框
\usepackage{amssymb}%数学符号
\usepackage{amsmath}%数学公式
\usepackage{booktabs}%设置三线表的线粗细
\usepackage{array}%table
\newcommand{\tabincell}[2]{\begin{tabular}{@{}#1@{}}#2\end{tabular}}
%\usepackage{cite}
% \ctexset{section={format={\zihao{3} \heiti \bfseries}},
% bibname={\zihao{-4} \heiti \bfseries 参考文献}}

\newcommand{\upcite}[1]{\textsuperscript{\textsuperscript{\cite{#1}}}}
\usepackage{caption}
\DeclareCaptionFont{heiti}{\heiti}
\captionsetup{labelsep=quad,font={small,bf,heiti},skip={4pt}}
\usepackage{geometry}
\title{\heiti \zihao{2} 神经机器阅读理解研究综述
\footnotetext{ 投稿日期: 2020-07-19 \\
\hspace*{1.8em}作者简介: 孙相会(1997-),男,硕士,研究方向为自然语言处理,E-mail: 2357094733@qq.com}
}%黑体2号 在标题那页插入脚注


\author{\zihao{-4} \songti 孙相会 \\ 东北大学 计算机科学与工程学院,沈阳 110169}
\date{}
%%%%%%%%%%%%%%%%%%%%%%%%%%%%%%%%%%%%%%%












\begin{document}
    \maketitle %生成title,author,date

        \noindent \textbf{摘\quad 要}: 
%机器阅读理解
%的目的是使得机器能够理解自然语言文本,
%它是自然语言处理领域
%十分重要的研究方向。
%随着深度学习技术的进步以及大规模数据集的发布,在机器阅读理解方向
%上的研究已经取得了很大的突破。近年来随着自然语言处理领域预训练模型的出现,再一次推动了机器阅读理解领域的发展。
%本文主要从四个方面对自2015年以来机器阅读理解领域的发展做综述:介绍机器阅读理解的任务定义以及相关数据集;
%分析机器阅读理解领域经典的基于注意力机制或推理结构的模型以及目前流行的预训练模型;
%探讨更加复杂的机器阅读理解任务;
%总结机器阅读理解领域目前存在的问题并且对未来的研究趋势做展望。\\
%\heiti 关键词: \songti 机器阅读理解;自然语言处理;预训练模型;注意力机制;推理结构 \\
%\heiti 中图分类号: \songti ? \hspace{1cm} \heiti 文献表示码: \songti A 
最后写\\
\textbf{文献标识码:} A  \qquad \textbf{中图分类号:} TP391
\begin{center}
    \textbf{\zihao{4} Overview of Studies on Neural Machine Reading Comprehension \\}

%     \zihao{5} Xianghui Sun \\
% \zihao{-5} id: 1971654 (NEU) 
%\zihao{-4} SUN Xianghui \\
%College of Computer Science and Engineering,Northeasten University,Shengyang 110169,China

\end{center}
\textbf{Abstract:} Machine Reading Comprehension aims to make machines comprehend the natural language documents, which 
is an important research direction
 in the field of natural language processing. With the development of deep learning technology and release of large scale datasets, the research on the field of machine reading comprehension has made great breakthroughs. 
 With the emergence of pre-trained model in natural language processing in recently years, it promotes the development of machine reading comprehension once again. This paper mainly make a survey from four aspects over the development of machine reading comprehension in recently years: to introduce the definition of 
  machine reading comprehension tasks and its corresponding datasets; to analyze classical model in the field of machine reading comprehension  
  which based on attention mechanism or reasoning structure as well as the currently 
  popular pre-trained model; to discuss more complicated machine reading comprehension tasks;
   to summarize the existing problem and look into the future research trend about machine reading comprehension. \\
\textbf{Key words:} machine reading comprehension; natural language processing; pre-trained model; attention mechanism; reasoning structure












%----------------------正文-----------------------
%\begin{multicols}{2}
\section{引言}

%\vspace{10cm}在垂直方向上,两行之间的距离
机器阅读理解(Machine Reading Comprehension,MRC)作为一项衡量计算机阅读理解文本能力的任务,是自然语言处理领域(Natural Language Processing,NLP)
十分重要也是具有挑战性的研究方向。通常情况下,MRC任务就是
给定一篇或多篇段落文本(passage),要求模型阅读这些段落后回答相关的问题(question)。
早期的MRC系统主要是基于规则和模式匹配的方法,或者通过概率统计的方式计算问题与文章之间的相似程度。这很难达到深层次的理解文本,
而且数据集规模比较小,系统难以获得期望的性能也不能实际的应用。
随着深度学习的兴起以及NLP领域出现的一些经典技术
如Word2Vec\upcite{word2vec}以及注意力(Attention)机制在NLP领域的应用\upcite{Bahdanau}等,这些技术的发展使得研究人员开始利用神经网络
构建机器阅读理解模型,
因此也叫神经机器阅读理解。

Hermann等人\upcite{Hermann}在2015年发布了规模比以往数据集都要大的阅读理解数据集CNN\&Daily Mail,并且提出两个基于神经网络和注意力机制构建的模型(Attentive Reader,Impatient Reader),这项工作可以视为机器阅读理解领域的奠基性工作。此后越来越多的学者在这两个模型的基础上构建效果更好的神经机器阅读理解模型,
Chen等人\upcite{AR}使用双线性函数以及Kadlec等人\upcite{ASR}采用点积运算的计算方式取代Attentive Reader的加法形式。Sordoni等人\upcite{IAReader}提出IA Reader模型利用循环神经网络的循环计算机制以及Dhingra等人\upcite{GAReader}提出的GA Reader模型通过加深网络的层次达到Impatient Reader模型多步推理的效果。

%而且规模越来越大,形式越来越复杂的数据集也相继发布。
SQuAD\upcite{SQuAD1}是由斯坦福大学在2016年发布的大规模数据集,问题的答案来源于原文中某一片段而不像CNN\& Daily Mail数据集仅仅是某个实体单词。
SQuAD数据集的发布极大地推动了MRC领域的发展,很多经典的神经机器阅读理解模型都是在SQuAD数据集上构建出来的。Wang等人\upcite{MatchLSTM}提出Match-LSTM模型,通过计算段落到问题的注意力将问题的语义融合到段落文本中,并且利用指针网络的机制预测答案在原文中的位置。在这之后很多模型受到Match-LSTM的启发,如Wang等人\upcite{RNet}在其基础上添加一层自注意力机制用来加强模型对段落的理解,Xiong等人\upcite{DCN}采用一种动态迭代指针网络的机制来多次迭代预测答案的位置。在2018年
斯坦福大学又发布了SQuAD 2.0\upcite{SQuAD2}数据集,在SQuAD的基础上增加了五万多个不可回答的问题。


在SQuAD发布不久后微软研究院发布了来源于真实场景下的数据集MS MARCO\upcite{MSmarco},数据集的问题来源于必应搜索日志上用户搜索的问题,从必应搜索的返回结果中选取10篇最相关的段落作为问题的答案依据,答案是人工生成的而不限于某段文本,因此数据集难度更大。类似的数据集还有TriviaQA\upcite{TriviaQA},DuReader\upcite{DuReader}等。针对这种多段落自由答案形式的任务,Tan等人\upcite{SNet}提出S-Net模型,先
通过片段抽取模块提取出一段文本作为答案的预测依据,然后利用生成模块生成答案。

然而以上大部分的数据集,与问题相关的答案通常集中在单个句子的局部上下文
这类数据集对模型的推理能力要求不高。
为了考察模型的推理能力,HotpotQA\upcite{HotpotQA},WIKIHOP\upcite{WIKIHOP},Narrative\upcite{NarrativeQA}等数据集相继发布,这些数据集中的问题要求模型从多个段落中逐步检索推理才能找到答案。此外还有多轮对话形式的阅读理解任务,相关的数据集如CoQA\upcite{CoQA},QuAC\upcite{QuAC}等。

每一个新的数据集都会在原有数据集的基础上增加各种各样的难度,从而不得不设计更加优秀的模型处理这些任
务,MRC领域也因此快速发展。
自2018年随着ELMo\upcite{ELMo}、GPT\upcite{GPT}、BERT\upcite{BERT}等预训练模型的出现,
再一次提升了机器阅读理解模型的性能,甚至在某些数据集上模型的表现超过人类水平。
%目前几乎所有数据集上表现最好的模型都是基于预训练模型的。

本文主要从MRC的具体任务概述出发,总共分5章,结构安排如下:
第2章介绍机器阅读理解的具体任务以及相应的评估指标;第3章
介绍神经机器阅读理解模型,包括经典的基于抽取式任务的MRC模型,复杂任务下的MRC模型以及基于预训练模型的MRC模型,对比它们的差异以及优缺点。
%同时介绍近年来NLP领域最受关注的预训练模型以及如何应用在MRC任务上。
第4章主要讨论MRC领域的发展历史和目前MRC领域存在的主要问题。
第5章对MRC领域做总结与展望。

% \end{multicols}
% %在你想要分栏的段落上下加上begin end columns{2}


																																	

\section{机器阅读理解任务概述}

%\footnotetext[1]{\url{www.cnn.com}}\label{1}

机器阅读理解任务是为了使得计算机具有对自然语言文本理解的能力,像人类一样阅读并且理解一篇文章。具体的就是给定
一篇文章$P$和一些与文章$P$相关的问题$Q$,
要求模型通过阅读$P$之后给出$Q$的正确答案$A$,即建模给定$P$和$Q$的条件下预测$A$的概率:
\begin{equation}
    P(A|P,Q)
\end{equation}
根据答案形式的不同,任务也是多种多样的,
大致可以概括为4类:完形填空、多项选择、片段选择和自由回答。
下面对这四种任务分别进行叙述并介绍相关的数据集。


% \begin{table*}[!ht]\tiny
%     \caption{CNN\&Daily Mail数据集中的一个样例}
%     %\centering
%     \vspace{10pt}
%     \resizebox{\textwidth}{!}{
%     \small
%     \begin{tabular}{l}
%         \hline
%         完型填空型数据集 \\
%         \toprule
%         \hspace{-13pt} \textbf{Context} \\
%         The BBC producer allegedly struck by Jeremy \\
% Clarkson will not press charges against the “Top \\
% Gear” host, his lawyer said Friday. Clarkson, who \\
% hosted one of the most-watched television shows \\
% in the world, was dropped by the BBC Wednesday \\
% after an internal investigation by the British broad- \\
% caster found he had subjected producer Oisin Tymon \\
% “to an unprovoked physical and verbal attack.” . . . \\
%         \midrule
%         \hspace{-13pt} \textbf{Query} \\
%         Producer X will not press charges against Jeremy \\
%         Clarkson, his lawyer says.\\
%         \midrule
%         \hspace{-13pt} \textbf{Anwser} \\
%         Oisin Tymon\\
% \bottomrule
        
%     \end{tabular}
%     }

% \end{table*}

% \begin{center}
%     \begin{table}\tiny
%         \caption{完形填空型数据集}
%     \begin{

%     \end{table}
% \end{center}














\subsection{完形填空}
完型填空型阅读理解是指给定一篇文章$P$和一个与文章相关的问题$Q$,$Q$是通过删除
掉句子中某一个单词构成,要求模型根据$P$能够
正确的填写出$Q$缺失的单词。CNN和Daily Mail数据集\upcite{Teaching Machines to Read and Comprehend}是
由Google DeepMind和牛津大学发布于2015年,
这是第一个较大规模的阅读理解型数据集。从CNN\footnote{www.cnn.com\label{cnn}}中收集93k篇文章,从
Daily Mail\footnote{www.dailymail.co.uk\label{daily mail}}上收集220k篇文章。每一篇文章
的作者都为这篇总结出一些具有概括性的句子,这些句子涵盖了这篇文章的要点。于是把这些概括性的句子删去其中的一个单词,
以此作为问题,构建了(文章-问题-答案)的三元组形式的语料库作为填空式的阅读理解任务。

% Children's Book Test(CBT)数据集\upcite{CBT}的构建是Facebook人工智能实验室bAbI
% 项目\footnote{https://research.fb.com/downloads/babi/\label{CBT}}的一部分。在CBT中所用到的书籍
% 来源于项目Project Guntenberg\footnote{https://www.gutenberg.org\label{book}}。对于书中的每篇文章,
% 选取21个连续的句子,前20个句子最为上下文段落,第21个句子挖掉其中的一个单词作为问题,给出10个单词作为候选答案,
% 要求模型从中选取正确的答案。
% CBT数据集与CNN/Daily Mail\upcite{Teaching Machines to Read and Comprehend}数据集构造的方式略有不同,
% CNN/Daily Mail中仅仅只将句子中的某个命名实体单词去掉,其余的命名实体单词用特殊的标记取代然后每次训练都要随机打乱,
% CBT一共去掉四种类型的单词,命名实体、常见的名词、动词、介词,从而形成四种类型的问题,对应着的答案也自然是四种类型。
% 同时对于每种类型的答案,从对应的文章和问题中选取出九个和这个答案一样词性的单词从而构成十个候选答案。

%\end{multicols}
% \begin{center}
%     \textbf{表1: 原始版本vs匿名版本}
%     \begin{tabular}{l l}
%         \toprule
%         Original Version&Anonymised Version \\
%         \midrule
%         \hspace{-13pt} \textbf{\zihao{4} Context} \\
%         The BBC producer allegedly struck by Jeremy&the ent381 producer allegedly struck by ent212 will \\
% Clarkson will not press charges against the “Top&not press charges against the “ ent153 ” host , his \\
% Gear” host, his lawyer said Friday. Clarkson, who&lawyer said friday . ent212 , who hosted one of the \\
% hosted one of the most-watched television shows&most - watched television shows in the world , was\\
% in the world, was dropped by the BBC Wednesday&dropped by the ent381 wednesday after an internal \\
% after an internal investigation by the British broad-&investigation by the ent180 broadcaster found he \\
% caster found he had subjected producer Oisin Tymon&had subjected producer ent193 “ to an unprovoked \\
% “to an unprovoked physical and verbal attack.” . . .&physical and verbal attack . ” . . . \\
%         \midrule
%         \hspace{-13pt} \textbf{\zihao{4} Query} \\
%         Producer X will not press charges against Jeremy&producer X will not press charges against ent212 ,\\
%         Clarkson, his lawyer says.&his lawyer says .\\
%         \midrule
%         \hspace{-13pt} \textbf{\zihao{4} Anwser} \\
%         Oisin Tymon&ent193\\
% \bottomrule
        
%     \end{tabular}
% \end{center}

% \begin{figure}
%     \centering
%     \includegraphics[width=0.2\linewidth]{CBT.png}
%     \caption{CBT数据集样例}
% \end{figure}
% \begin{multicols}{2}
\subsection{多项选择}
多项选择型这类问答任务是对于给定的文$P$,以及和$P$相关的问题$Q$和多个
候选答案$A=\{A_1,A_2,\cdots,A_n\}$,
从中选择正确的答案,即$P(A_i|P,Q)$,其中$A_i \in A$。
相关的数据集如RACE\footnote{www.cs.cmu.edu/glail/data/race/\label{race}},这个数据集是从中
国中学生的英语考试题中
建立的数据集。共有将近28000篇文章以及100000个问题,答案并不是简单的限制于
文章中的单词,而且答案和问题中单词可能
从没有在文中出现过,因此简单的利用单词匹配方式并不能达到很好的效果。
这些问题和候选的答案都是由出题专家生成的,因此更加的
接近真实世界的语义。另外RACE数据集中文章主题的覆盖度比其它的数据集更广泛,比如CNN/Daily\upcite{Teaching Machines to Read and Comprehend}
所有文章全都是来源于CNN新闻,SQuAD\upcite{SQuAD1}数据集所有的文章全都是来源于维基百科。
而RACE数据集涵盖多个领域如新闻、故事、广告、传记等等。由于其类型的多样性因此可以更好的评估机器的阅读理解
能力。

% \begin{wrapfigure}[15]{r}[\dimexpr 1\columnwidth+\columnsep\relax]{0.5\textwidth}%[htbp]
%     \centering
%     \includegraphics[width=0.5\textwidth]{race.png}
%     \caption{RACE数据集样例\upcite{RACE}}
% \end{wrapfigure}


\subsection{片段选择}
这类问答任务是MRC领域较为流行的研究方向,给出$P$和问题$Q$,问题的答案是$P$中的一段连续的单词构成,
答案的长度不固定,可以表示为$P(A|P,Q)$,其中$A=\{t_i,t_{i+1},\cdots,t_{i+k}\}(1\leq i\leq i+k\leq n)$,$n$代表$P$中
单词的个数。抽取式问答任务最为
广泛使用的数据集是SQuAD 1.1\upcite{SQuAD1}和SQuAD 2.0\upcite{SQuAD2}。其中
SQuAD 1.1是Stanford问答数据集的第一个版本,
由众包工人在维基百科上面的文章中给出问题,答案来源于文章中某段连续的单词。
SQuAD 1.1含有536篇文章,总计107785个问题-答案对。
在SQuAD 2.0中又在原有的数据集中加入了50000多个没有答案的问题,准确的说
这些问题的答案不在相应的文章中。

%\end{multicols}


% \begin{figure}[htbp]
%     \centering
%     \includegraphics[width=0.5\linewidth]{race.png}
%     \caption{RACE数据集样例\upcite{RACE}}
% \end{figure}

% \begin{figure}
%     \includegraphics{squadv1.png}
%     \caption{SQuAD v1.0}
% \end{figure}

% \begin{figure}
%     \includegraphics{squad2.png}
%     \caption{SQuAD v2.0}
% \end{figure}


%\begin{multicols}{2}
\subsection{自由回答}
简单的从文章中摘取一段文本可能并不能回答问题需要的答案,自由回答型任务的答案是自由形式的
,不局限于文章中的某些单词,语法上往往是更加的灵活。
可以表示为$P(A|P,Q)$,其中$A\subseteq P$或$A\nsubseteq P$。
从文章中概括提炼出问题的答案也是更加的符合人类的阅读方式的,基于这些原因,自由回答式问答的数据集
也因此公布出来并且受到广泛的关注,
相关的数据集如MS MARCO\upcite{MS marco}。MS MARCO是由微软通过在必应搜素引擎的日志上收集用户
提出的问题,对于文本段落是来源于必应搜素引擎的返回的搜索结果。具体的就是对于每一个问题,给出10个最相关的
查询结果的文本段落,然后由标注人员从这10个文本段落中找出那些与这个问题有关的文本段落,然后人工的从这些选择出来的段落中
概括提炼出答案,同时对于选出来的段落要标记为$is\_select=1$,表示这个段落和答案相关,从而可
以训练模型。

%\vspace{15\baselineskip}
因此可以看到
这个数据集与前面的数据集很大的不同之处就是答案是人工
生成的,不局限于文本段落中固定的一段单词,因此更加的接近现实世界中的人类阅读理解,
对模型的推理能力要求也更高。同时如果不能从给出的10个段落中推理出答案,
那么这个问题就标记为不可回答的问题,同样要保留在数据集中,
目的就是让模型能够判别出问题是否具有答案。
% 如图4为MS MARCO数据集的一个样例。
% %\end{multicols}

% % \begin{wrapfigure}[10]{r}[0em]{0.5\textwidth}%[htbp]
% %     \centering
% %     \includegraphics[width=0.5\textwidth]{race.png}
% %     \caption{RACE数据集样例\upcite{RACE}}
% % \end{wrapfigure}

% %\vspace*{5\baselineskip}

\begin{table*}[!ht]
    \centering
    %表格超出页边距用resizebox
    \resizebox{\textwidth}{!}{
    \begin{tabular}{l l c}
        \hline
        填空型数据集&& \\
        \hline
        \multirow{3}{*}{CNN\&Daily Mail}& 文章:&22222222\\
        \cmidrule(l){2-3}
        &问题:&3333 \\
        \cmidrule(l){2-3}
        &答案:&4444 \\
        \hline
    \end{tabular}
    }
\end{table*}

% \begin{figure}
%     \centering
%     \includegraphics[width=0.8\linewidth]{msmcro.png}
%     \caption{MS marco样例}
% \end{figure}
% \subsection{对话型问答}
% 抽取式问答尽管比填空式问答更复杂,需要模型更强的推理能力,但是这类任务要求答案来源于文章中的一段文本,这样的答案往往不符合现实世界中
% 人与人之间的那种用很短的句子或者单词来回答问题的交互方式。另外简单的从段落中提取出一短文本可能不能回答问题,例如图中的问题4$Q_4$(How many?)。
% 文献\upcite{CoQA}公布了一个用来做对话式问答的数据集,数据集有来源于8千个对话文章的1.26万问题和对应的答案,这些文章取自
% 七种不同的领域包括新闻、考试、维基百科等语料库中收集的文本段落,
% 由提问者给出问题以及回答者给出答案同时给出答案所在的句子或者段落以供模型学习推理,如图所示。可以看出在CoQA数据集中模型必须理解文本段落以及一系列的问题才能给出正确的答案,
% 因为对话中的问题依赖于前面提出的问题和给出的答案,如图中问题5$Q_5$,问题仅仅是Who?所以模型必须理解前面对话中的问题和答案。
% 这种任务的答案是自由式的,不限于任何的文本段落,同样是自由式答案的数据集还有文献\upcite{MS marco},不同的是在CoQA任务中,训练的时候模型要
% 先从文本段落中选出来包含有答案的段落或句子,然后总结概括生成语法形式自由的答案,测试时直接给出答案即可。



% \begin{center}
%     \textbf{Table 1 Anwser type distribution in SQuAD} \\
%     \vspace{5pt}%vspace 是垂直方向拉开距离
%     \begin{tabular}{l l  l}%p{} 设置列宽
%         \toprule
%         Anwser type&Percentage&Example \\
%         \midrule
%         Date&8.9\% &19 Octiber 1512 \\
% Other Numeric&10.9\% &12 \\
% Person&12.9\%&Thomas Coke\\
% Location&4.4\%&Germany\\
% Other Entity&15.3\%&ABC Sports\\
% Common Noun Phrase&31.8\%&property damage\\
% Adjective Phrase&3.9\%&second-largest\\
% Verb Phrase&5.5\%&returned to Earth\\
% Clause&3.7\%&to avoid trivialization\\
% Other&2.7\%&quietly\\
%         \bottomrule
%     \end{tabular}
% \end{center}
% %\vspace{10pt}%vspace 是垂直方向拉开距离
% 从表中可以看出答案的形式是多样的,不仅仅只是单一的名词实体等。至于问题的难度,
% 或者说为了能够得到答案模型所需要的推理能力,作者通过在验证集上的48篇文章中选出来192个问题通过手工的
% 标注出回答每个问题所需要的推理能力,结果表明,每个问题与答案之间都至少存在词汇或语法上的变体,
% 例如同义词变换,多句子推理等。

\subsection{评估方法}
对于不同的MRC任务有不同的评估指标。
对于填空型任务与多项选择型任务都是属于客观题型,用准确率就可以衡量模型的性能。
片段选择型任务属于半客观题型,通常用精确匹配EM(Exact Match)和F1值来评估模型,F1值是精确率和召回率
之间的调和平均数。对于自由回答式任务的答案,一般采用单词水平的匹配率作为评分标准,常用标准有
ROUGE\upcite{ROUGE}。下面详细介绍这几种评估指标如何评估不同的MRC任务。
\subsubsection{准确率}
准确率可以用来评估完形填空和多项选择这两种类型的任务。
对于测试集合中的所有问题$Q=\{Q_1,Q_2,\cdots,Q_m\}$,其中$m$代表问题的个数。如果模型预测出来的
$m$个答案中有$n$个是正确的,那么模型的准确率自然是$n/m$。
精确匹配EM评估指标可以看做是准确率的扩展,就片段选择型任务来讲,EM要求
预测出来的所有单词要和标准答案
的所有单词要精确匹配下,EM值才为1,否则为0。
因此最后模型在测试集上的EM值也就是$n/m$。
\subsubsection{F1}
F1分数是最为普遍使用的一种评估标准,不仅仅局限于MRC的各种任务。
F1值是精确率(precision)和召回率(recall)
之间的调和平均数。
具体计算公式如下:
\begin{equation}
    \text{F1}=\displaystyle\frac{2\times\text{precision}\times\text{recall}}{\text{precision}+\text{recall}}
\end{equation}
精确率是指模型预测的答案中有多大比例的单词是标准答案中的单词。
召回率是指标准答案中的单词有多大比例在预测答案中出现。

\subsubsection{ROUGE}
ROUGE(Recall-Oriented Understudy for Gisting Evaluation)最初是用来
评估生成文本摘要的一种方法,因为可以ROUGE的计算机制
来评估MRC领域中自由答案型任务。ROUGE的评分有多种,在MRC领域较为常用的是ROUGE-L,
ROUGE-L用来计算标准答案和预测答案的最长公共子序列(Longest Common Subsequence,LCS),ROUGE-L的计算
公式如下:
\begin{gather}
    R_{LCS}=\displaystyle\frac{LCS(X,Y)}{m} \\
    P_{LCS}=\displaystyle\frac{LCS(X,Y)}{n} \\
    F_{LCS}=\displaystyle\frac{{(1+\beta)}^2R_{LCS}P_{LCS}}{R_{LCS}+\beta^2P_{LCS}}
\end{gather}
其中$LCS(X,Y)$表示标准答案与预测答案之间的最长公共子序列,
$m$和$n$分别代表标准答案和预测答案中单词的个数,$\beta$是ROUGE-L的的参数,
用来控制精确率和召回率的重要程度。










\section{神经机器阅读理解模型}
%随着大规模机器阅读理解数据集如CNN\&Daily Mail\upcite{CNNDailyMail},SQuAD\upcite{SQuAD1}等的发布以及
%深度学习技术的发展,神经机器阅读理解模型的性能显著的超过传统的基于规则和特征的模型,随着NLP领域预训练模型的发展,基于预训练模型
%来做MRC任务的模型性能再一次的提升。
Hermann等人\upcite{Hermann}发布的CNN\&Daily Mail数据集以及他们所设计的两个基于神经网络和注意力机制的模型可以看作是MRC领域的奠基性工作,开创了神经机器阅读理解模型。Rajpurkar等人\cite{SQuAD1}在2016年发布的SQuAD数据集是MRC领域里程碑式的数据集,在2016-2018年期间掀起了一阵热潮,很多的神经机器阅读理解模型都是在此期间构建出来的。
填空式数据集本质上可以认为是抽取式数据集的简化形式,而后续的很多任务如对话形式、开放领域形式、多段落形式的阅读理解任务也都是在抽取式任务的形式上设计模型。因此抽取式阅读理解任务是MRC领域的核心,本章主要以抽取式任务的MRC模型为出发点,
安排如下:3.1节分析经典的基于抽取式任务的MRC模型通用架构,3.2节介绍复杂任务下的MRC模型,3.3节介绍目前流行的预训练模型以及如何利用预训练模型设计性能更强大的MRC模型。
\subsection{基于抽取式任务的经典MRC模型}
想让机器能够阅读理解文本需要解决以下几个问题:
\begin{enumerate}
	\item 如何将段落和问题这种文本形式的无结构数据表示为计算机可以处理的形式;
	\item 如何根据问题检索出段落中与问题最相关的部分;
	\item 如何从检索出来的文章片段中归纳得到答案。
\end{enumerate}
用于MRC任务的深度学习模型的整体框架主要包括如下几个层:词嵌入层、编码层、交互层、答案输出层,如图1所示。
词嵌入层的作用是将段落和问题嵌入到低维的向量空间中,用每一个向量表示每一个单词。
编码层的作用是编码段落和问题中单词的语义信息,使得每一个单词可以关注到它的上下文。交互层的作用是
将段落的语义信息与问题的语义信息融合,让模型学习到段落中与问题最相关的部分。
答案输出层的作用是从段落中查找出问题的答案。
\begin{figure}
	\centering
	\includegraphics[width=8cm,height=7cm]{generic.png}
\end{figure}

\subsubsection{词嵌入层}
如何将文本有效的表示成计算机可以处理的形式同时可以有效地利用单词之间的语义一直的NLP领域
的重点问题。早期的one-hot形式编码用一个二值向量表示单词,但是存在数据稀疏并且随着单词个数的增加出现维度灾难的问题,此外这种形式的编码也不能够表示出单词之间的语义关系。

Rumelhart等人\upcite{Rumelhart}最早提出分布式表示的概念,
分布式表示是将单词用一个低维度的稠密向量表示,即将单词嵌入到一个低维向量空间中,因此这种表示方式也叫词嵌入。语义相近的单词在向量空间中距离也相近,因此这种词表示方法解决了one-hot编码的很多问题。
Bengio等人\cite{NNLM}最早将深度学习的思想融入到语言模型中提出神经网络语言模型(Neural Network Language Model,NNLM)模型,模型的第一层映射矩阵就是学习到的词向量,Mikolov等人\upcite{word2vec}受到这种思想的启发提出Word2Vec。Word2Vec提出两种模型CBOW和Skip-gram来学习单词的分布式表示,CBOW使用中心词的上下文来预测这个单词而Skip-gram利用中心词来预测其周围的单词。但是无论是CBOW还是Skip-gram都只是考虑了单词局部上下文的信息,GloVe\upcite{GloVe}利用单词共现矩阵考虑了全局统计信息。


%最流行的生成分布式词向量的技术如Word2Vec\upcite{word2vec}和GloVe\upcite{GloVe}。
大量实验表明利用Word2Vec或者GloVe预训练好的词向量作为下游任务文本的词表征来初始化下游任务模型的第一层可以显著地提升模型的效果。
除了词嵌入方法外,还有很多细粒度的嵌入方式。如Seo等人\upcite{BiDAF}提出在词嵌入的基础上结合单词的字符嵌入,以缓解NLP领域常见的OOV(out-of-vocabulary)问题。Chen等人\upcite{DrQA}提出引入单词的语义特征来增强嵌入表示,如段落单词与问题单词之间的完全匹配特征、词性特征以及单词的命名实体特征等。

然而Word2Vec和GloVe训练出来的词向量是
静态的词向量,即训练好模型后一个单词的表示
向量就是固定的,没有考虑上下文的信息,因此无法解决多义词问题。为了解决这个问题,Peters等人\cite{ELMo}提出一种动态的基于上下文的词嵌入模型ELMo,每一个单词的词向量都是根据它所在的上下文语义表示的,很好的解决了一词多义的问题。关于ELMo以及预训练模型的细节见\ref{pretrain}节。

从早期的one-hot形式编码到分布式表示技术最后到基于上下文的词嵌入技术,每一种技术的出现都证明了一个好的文本表示方法可以
极大地提升模型的性能。
\subsubsection{编码层}
这一层的目的是在词嵌入层的基础上通过对词嵌入层的输入文本做特征提取,进一步获得句子层面的语义信息。
NLP领域最为常用的特征提取器
是基于循环神经网络(RNNs)的变体如LSTM\upcite{LSTM}和GRU\upcite{GRU}等,因为这种循环结构适合处理文本这类序列数据,绝大部分的MRC模型编码层都是利用RNNs作为特征提取器。
但也正是这种序列式的结构使得计算不能并行,训练耗时,更重要的是由于梯度消失所以不能解决单词之间长距离依赖问题,使得其
特征提取能力始终受限。Vaswani等人\upcite{Transformer}提出了一种用于机器翻译的encoder-decoder结构transformer,舍弃了常用的循环神经网络结构,完全的基于自注意力机制构建模型,实验表明transformer的特征提取能力强于循环神经网络而且可以并行计算加快训练。
文献\cite{QANet}提出一种网络模型QANet,不像之前的那些模型几乎都是用RNNs来做编码器,
QANet提出一种新颖的编码结构,利用卷积结合transformer\upcite{Transformer}中的
多头注意力结构
。
卷积方式采用的是文献\cite{DSC}提出的深度可分离卷积(depthwise 
separable convolutions),
相比传统的卷积计算方式深度可分离卷积可以减少运算次数。
整个结构的思想是先利用卷积操作建模局部特征的交互,再用自注意力机制建模全局交互,
实验结果表明这种架构不仅加快训练速度同时在SQuAD数据集上模型性能优于那些利用RNNs作为编码器的模型。
关于transformer的细节介绍见\ref{transformer}节。


%每一层由多头自注意力和前馈网络(FN)构成。Transformer的整体架构
%通过利用自注意力(self-attention)机制取代RNN那种序列式的计算方式,
%对于一个句子中的两个单词不考虑单词之间顺序的关系,直接计算它们之间的相关度,例如计算两个单词向量表示的内积。
%自注意力机制可以捕获句子中长距离依赖的特征关系,
%解决了循环神经网络固有的序列式传递信息导致后面的单词与前面的单词
%之间达不到有效的信息传递问题。
%通过自注意力机制不仅可以做到
%单词之间的全局交互同时其并行计算使得模型训练时间大幅减少。Transformer的encoder端和decoder端都可以做特征提取器如BERT\upcite{BERT}用encoder端特征提取,GPT\upcite{GPT}用decoder端特征提取,实验证明在大规模数据集上transformer的特征提取能力要强于
%基于RNNs\footnote{用RNNs来统一表示RNN的变体,如LSTM,GRU等\label{RNNs}}的编码器,目前几乎所有的NLP预训练模型都是利用transformer作为特征提取器。

\subsubsection{交互层}
%在预测答案时需要将问题的语义信息与文章的语义信息关联,这样模型在预测答案时才能知道文章中哪一部分是问题的答案。
%通常利用注意力机制实现这一目的,注意力机制就是让模型关注到重点的部分,不同的注意力计算方式很大程度上影响模型性能,
%后面将详细介绍基于注意力机制的模型以及它们不同的计算方式。
交互层是整个网络模型中关键的一层,前面的编码层输出的是问题和文章中每个单词的上下文语义编码,每个单词
关注了自己所在句子的上下文单词,但是却并没有关注对应的句子。而我们在做阅读理解问题时,通常是带着
问题去文章中找答案,我们要知道文章中每一个单词和问题之间的相关度。因此交互层的目的就是让文章的语义信息与问题的
语义信息融合,以此达到对文章更深层次的理解,而交互层中最常用的方法就是注意力机制。

注意力机制可以被视为是一个查询向量(query)和一组键值对向量(key-value pairs)的映射过程。整个过程首先是利用函数$f$衡量query和key之间的相似度,生成一个权重分数向量,然后将权重分数向量归一化(通常利用softmax函数)后对value加权求和得到的结果就是query对key-value pairs的注意力。具体计算公式形式如下:

\begin{gather}
\alpha_i=\text{softmax}(f(Q,K_i)) \notag \\
\text{Attention}(Q,K,V)=\sum_{i=1}^{n}\alpha_iV_i
\end{gather}
其中$(K_i,V_i)$代表key-value pairs中的第$i$个值,
函数$f$常采用计算方式有内积函数、二次型函数、
前馈神经网络,双维度转换函数,分别见如下公式:
\begin{gather}
f(p_i,Q)=p_i^TQ \qquad \text{内积函数} \\
f(p_i,Q)=p_i^TWQ\qquad \text{二次型函数}\\
f(p_i,Q)=v^T\tanh(Wp_i+UQ)\qquad \text{前馈神经网络} \\
f(p_i,Q)=p_i^TW^TUQ \qquad \text{双维度转换函数}
\end{gather}

在NLP领域中$K=V$,简单的来说就是两个序列中其中一个序列为另一个序列的每一个位置生成一个权重值,这个值代表当前位置的单词对另一个序列的重要性。
如果是自注意力(self attention),那么此时$Q=K=V$,目的是计算序列中某个单词和其它单词之间的相关性从而增强自身的语义表示。
Bahdanau等人\upcite{Bahdanau}最早将
注意力机制应用在机器翻译领域,获得了极大的反响,
为NLP领域的其它任务的模型提供了启发式的思想。

MRC模型做交互注意力运算有两个方向,即从问题到段落(Question-to-Context,Q2C),从段落到问题(Context-to-Question,C2Q)这两个方向。从Q2C的注意力是指
将问题看做是$Q$,段落看做$K,V$。利用问题去和文章做注意力计算,
%得到问题对文章的注意力权值,它代表问题对文章每一个单词的关注程度,利用它对段落的表示向量加权求和得到的就是问题的一个新的表示向量。
%以Q2C的注意力计算过程为例,
定义$C=[c_1,c_2,\cdots,c_n] \in R^{n\times d}$代表段落的表示向量,其中$n$代表段落长度,$d$代表向量维度,
$Q\in R^{d}$代表整个问题的表示向量
%每一个
%$p_i$代表文章中单词的语义向量表示,$Q$代表整个问题的语义信息。
Q2C的注意力计算步骤如下:
\begin{gather}
\alpha_i=\text{softmax}(f(c_i,Q)) \notag \\
\text{Attention}(C,Q)=\sum_{i=1}^{n}\alpha_{i}c_i \notag 
\end{gather}
%就是利用问题
%的语义信息和文章中每一个单词的向量表示计算它们之间的相关性,最后得到一组注意力权重$\alpha=[\alpha_1,\alpha_2,\cdots,\alpha_n]$
%表示文章与问题之间的相关程度,利用$\alpha$对文章加权求和就可以提取出来文章中与问题最相关的单词,
%这些单词对于回答问题是至关重要的。
C2Q注意力类似,此时段落看作是$Q$,问题看做$K,V$。
%拿段落和问题做注意力计算,段落的每一个单词都会
%关注到问题,然后利用注意力权值对问题加权求和从而计算得到段落的新的表示向量。

上面的式子是将问题压缩成一个固定维度的向量,得到的注意力权重$\alpha$也是一维的,因此也称为一维注意力。
一维注意力方法所关注的是问题序列的整体对文章的注意力,没有考虑问题序列的不同单词之间对文章的关注程度差异。
与其相对应的是二维注意力,即对于问题序列中的
每一个单词都会和段落做注意力计算,得到的注意力权重是二维向量。

%段落到问题注意力可以看做是带着文章阅读问题,问题到段落则可以看做是带着问题阅读文章。
C2Q和Q2C这两种都属于交互的计算注意力,然而这种注意力机制可能导致只重视文章中与问题相关度高的单词,而忽视了文章所强调自身的语义信息。在文章上利用自注意力机制则可以看做是反复的阅读文章,从而加深对文章语义信息的理解。








如果按照注意力的计算次数上区分,
又可以分为one-hop和multi-hop形式。one-hop,也叫“单跳结构”是指仅仅通过一次计算得到注意力权值然后加权求和得到注意力结果,这也是一种
静态的计算形式。与之对应的是multi-hop,也叫“多跳结构”。one-hop形式下仅仅只做一次交互计算,而注意力机制虽然可以提取相关的重要信息,但是
它仍然是基于浅层语义信息的相似度计算。在机器阅读理解任务中,
对于复杂的问题通常是不能在一个句子中找出答案,需要多步推理才能寻找答案,如表6所示
%\begin{figure}[ht]
%	\centering
%	\includegraphics[width=0.5\textwidth]{end2end.png}
%	\caption{MemN2N\upcite{MemN2N}提出的多步推理的一个样例 \\ Figure 1 An example of multi-hop reasoning from MemN2N}
%\end{figure}

\begin{table}[ht]
	\centering
	\caption{WIKIHOP\upcite{WIKIHOP}多跳推理的样例}
    \begin{tabular}{l p{15.0cm}<{\raggedright}}
	\toprule
	文章1:&\tabincell{l}{The Hanging Gardens, in \textcolor{blue}{[Mumbai]}, also known as Pherozeshah
		Mehta Gardens, are terraced gardens …  \\
		They provide sunset views
		over the \textcolor{red}{[Arabian Sea]} …} \\

	文章2:&\tabincell{l}{\textcolor{blue}{Mumbai} (also known as Bombay, the official name until 1995) is the
		capital city of the Indian state of Maharashtra. \\It is the most
		populous city in \textcolor{green}{India} …}\\
	%\cmidrule(l){2-3}
	文章10:&\tabincell{l}{The \textcolor{red}{Arabian Sea} is a region of the northern Indian Ocean bounded
		on the north by \textcolor{magenta}{Pakistan} and \textcolor{magenta}{Iran}, \\ on the west by northeastern
		\textcolor{magenta}{Somalia} and the Arabian Peninsula, and on the east by India …}\\
	\hline
	问题:&Hanging gardens of Mumbai, country,? \\
	%\cmidrule(l){2-3}
	\midrule
	答案:&{Iran, \textcolor{green}{India}, Pakistan, Somalia} \\
	\bottomrule
\end{tabular}
\end{table}                     
% \begin{figure*}
%     \includegraphics[]{multihop.png}
% \end{figure*}
%当前时刻计算的注意力结果要保留到下一时刻,即每一时刻都要计算注意力,是一种动态的计算形式,典型的代表是Bahdanau注意力\cite{neural machine translation by jointly learning to align and translate}。
我们可以看到想要得到最终的答案需要在多个段落中进行多次推理,在每一个推理过程中都会变换注意力关注的对象,显然one-hop结构是不能实现多步推理的。

鉴于目前多数模型交互层所使用的注意力机制较为复杂,很难按照上述形式完全的区分开每一个模型,本文按照Liu等人\cite{Survey}的思路按照注意力计算的方向以及次数划分各个模型。
\vspace{1ex}

\noindent\textbf{单向注意力} \quad Hermann等人\cite{Hermann}最早利用神经网络模型并且融入注意力机制做MRC任务。
文中提出两种不同的单向注意力机制Attentive Reader
和Impatient Reader,均是计算问题到文章的注意力。
Attentive Reader是将问题表示为一个固定长度的向量然后与文章中每一个单词做注意力计算,然后利用注意力权重对文章
中的单词的向量表示加权求和得到一个固定长度的
向量即为注意力运算后的结果,然后与问题联合预测答案,其中注意力的运算方式采用前馈神经网络(公式4)。
%可见这种注意力的计算方式仅仅只计算一次,因此属于one-hop类型。
Chen等人\cite{AR}在Attentive Reader的基础上利用双线性项(公式3)取代原有的前馈神经网络(公式4)并且直接将
对段落加权求和后得到的向量作为预测答案的输入而不是联合问题$Q$的语义信息,实验证明这种简化反而提高了模型的准确度。文献\cite{MatchLSTM}提出一种Match-LSTM模型,
与之前的模型不同,Match-LSTM计算的方向是段落到问题的注意力,将问题的语义信息融入到Match-LSTM中。
具体计算过程如下:
\begin{gather}
s_t=v^T\tanh(W^QH^Q+W^Ph_t^P+W_rh_{t-1}^r) \notag \\
\alpha_{t}=\text{softmax}(s_t)
%c_t=\sum_{i=1}^{m}a_i^tu_i^Q
\end{gather}
其中$H^Q$是问题通过编码层的输出,$h_t^P$是段落的第$t$个单词通过编码层的输出,$h_{t-1}^r$是Match-LSTM上一时刻
的隐藏状态。
$\alpha_{t}$是
段落的第$t$个单词与问题的每一个单词之间的注意力权重。
计算得到的注意力权重对问题的语义表示加权求和,然后与段落当前时刻单词的上下文表示拼接作为
Match-LSTM当前时刻的输入。
\begin{gather}
z_t=[h_t^p;\alpha_tH^q] \notag \\
h_{t}^r=\text{LSTM}(z_t,h_{t-1}^r)
\end{gather}
此外为了使得段落从后向前的对问题做关注,模型将段落序列翻转再次按照上述方式计算,最后将两个方向的计算结果
拼接作为交互层的输出。
\vspace{1ex}

\noindent \textbf{双向注意力} \quad 上述的模型全都属于单方向注意力,要么仅计算C2Q方向的注意力或者仅计算Q2C方向的注意力。
Xiong等人\cite{DCN}提出Dynamic Co-attention Network(DCN)模型,
在交互层中采用协同注意力机制,
协同注意力同步的计算文章对问题的注意力以及问题对文章的注意力。
最后按照公式(8)融合两个方向的注意力作为交互层的输出。
\begin{equation}
\widetilde{C}=\beta[Q,\alpha C]
\end{equation}
其中$\alpha$和$\beta$分别表示段落与问题之间的注意力权重,$\widetilde{C}$同时融合了问题的语义信息和段落的语义信息。
文献\cite{BiDAF}提出(Bidirectionl Attention Flow,BiDAF)模型。
同样计算两个方向(C2Q和Q2C)的注意力,但是与之前模型不同的是BiDAF将之前的段落语义表示和交互层计算得到的问题感知的段落语义表示一起流向后面的层,这样一定程度上避免了过早的对段落语义信息概括而导致
信息的损失。模型的简化实验表明C2Q方向的注意力对模型的重要性大于Q2C方向的注意力,一种可能的原因是由于问题序列的长度小于段落文本的长度所以计算得到的段落感知的问题语义向量的信息不够充分。
\vspace{1ex}

\noindent \textbf{单跳结构} \quad 单跳
结构是指段落与问题仅仅通过一次计算得到注意力权值然后加权求和得到注意力结果,要么是将问题整体压缩为一个向量与段落计算一次注意力,如Attentive Reader\upcite{Hermann},AS Reader\upcite{ASR}等,或者问题与段落的的整体表示采用并行化的计算方式,如DCN\upcite{DCN},BiDAF\upcite{BiDAF},QANet\upcite{QANet}等。
\vspace{1ex}

\noindent \textbf{多跳结构} \quad 多跳结构可以视为单跳结构的堆叠,目的是通过多次计算段落与问题的交互信息加深模型对段落和问题的理解,从而达到多步推理的目的。
实现多步推理这种机制通常有以下几种方式:
\vspace{1ex}

\noindent 第一种方式是基于之前时间步所计算得到的问题感知的段落语义信息计算下一时间步的段落和问题交互,如Impatient Reader\upcite{Hermann},并不是像Attentive Reader将
	问题表示为一个固定长度的向量,而是对于问题中的每一个单词都要和整个段落做注意力计算,而且计算的结果
	要和下一个单词以及段落共同做注意力计算,最后一个单词的注意力结果作为整个Impatient Reader计算注意力过程的输出。
	这种方式类似于人在阅读过程中不断的在问题和文章之间做关注。
\vspace{1ex}

\noindent
第二种方式是利用RNNs这种基于上一时刻隐藏状态更新下一时刻隐藏状态的循环特性来达到多步推理,Sordoni等人\upcite{IAReader}提出Iterative Attention Reader(IA Reader)模型,利用BiGRU存储每一次
迭代计算得到的问题和段落的交互信息。在每一时间步上,首先利用上一次的BiGRU的状态与问题做一维注意力匹配提取出问题的语义信息,
然后再结合上一次的BiGRU的状态与段落再做一维注意力匹配从而提取出段落的语义信息。将问题与段落的语义信息
通过各自的门控单元作为BiGRU当前时刻的输入,其中门控单元采用前馈神经网络用来解决当前时间步下问题和段落的语义信息提取不充分的问题。
%基于注意力机制的模型大部分是利用RNN的循环机制来存储每次交互的状态,
%但是由于RNN的梯度消失问题可能会丢失语义信息,因此文献\cite{memory network}提出一种记忆网络架构(memory network),
%通过外部记忆模块来存储语义信息,但是这种记忆网络并不能端到端的训练。
%为了处理这个问题,文献\cite{MemN2N}提出一种端到端的记忆网络(MemN2N)。
%利用记忆槽存储文档中每一个句子的嵌入矩阵,记忆槽的状态以及问题的语义信息会随着与文档的多次交互不断更新。
Shen等人\upcite{Reasonet}提出一种动态决定推理次数的模型ReasoNet,不同于IA Reader模型在整个推理过程中有着固定的推理次数。这种固定推理次数的缺点就是不考虑问题的复杂性,
对于复杂的问题往往需要模型多次的推理,因此不同题目难度需要不同的推理次数,应当让模型学会什么时候终止推理。为了达到这一目的,
ReasoNet模型利用一个终止门产生二元值输出来动态的决定是否继续推理。ReasoNet模型大致分为外部记忆单元模块、内部控制器模块、终止门模块以及答案输出模块。
具体的,将段落和问题通过Bi-GRU编码后的语义表示作为
外部的记忆单元$M$,利用内部控制器(采用GRU)当前时刻的状态与$M$做二维注意力匹配,得到注意力结果输入到内部控制器中
更新内部控制器的状态。终止门模块以当前时刻内部注意力的状态作为输入来判断是否需要继续推理。由于产生了二元离散输出值,
使得模型不能用梯度下降法训练,因此模型引入强化学习机制训练。
%其它的multi-hop结构的模型如Match-LSTM\upcite{MatchLSTM},R-Net\upcite{RNet},等。
\vspace{1ex}
%

\noindent
第三种方式通过堆叠多个计算注意力的层数达到多步推理的目的,
Dhingra等人\cite{GAReader}提出Gated Attention Reader(GA Reader)模型,类似于IA Reader模型,同样采用
BiGRU作为编码模块实现多跳结构。在每一步的推理过程中,首先通过BiGRU得到问题的语义信息,然后对段落的
每一个单词做注意力的计算得到问题感知的段落表示,同时采用点乘计算的门控机制建模问题感知的段落表示和原来的段落语义向量之间的交互关系,目的是利用问题更新文章的语义表示。这种处理过程
类比于带着问题反复的阅读文章,每一次都加深对文章的语义理解。
Wang等人\cite{RNet}提出一种带有门控机制的注意力循环神经网络以及自注意力机制联合的交互层设计模型RNet。RNet在交互层的设计分为两部分。
第一部分是带有门控机制的注意力循环神经网络,整体计算
方式类似于Match-LSTM,而且额外加入了门控机制使得模型可以有选择的输出
语义信息。具体的,在公式(7)中的$z_t$上添加一个门控单元:
\begin{gather}
g_t=\text{sigmoid}(W_gz_t)\notag \\
z_t^{*}=g_t\odot z_t
\end{gather}
其中$\odot$表示元素之间的点乘。
%由于$z_t=[h_t^p;\alpha_tH^q]$,$h_t^p$表示的是文章的第$t$个单词的语义表示,$\alpha_tH^q$表示的是
%对问题语义表示的融合,因此
通过添加门控单元使得模型可以有选择的决定哪部分作为重要的语义信息输出。这种机制类似于人在阅读过程中要
忽略段落中那些与问题无关的信息,凸显出重要的信息才能更加准确的找到答案。
第二部分是利用自注意机制对段落的语义信息再次交互建模,
%基于注意力机制的循环神经网络的输出对关注了问题的
%文章语义表示与原始文章语义表示建模后的输出,而这种计算机制的问题之一是
%两个距离较远的单词之间交互信息由于梯度消失等原因会变得很弱。因此
通过自注意机制可以使得段落中每一个单词关注到其余所有的单词,使得模型对段落达到更深层次的理解。
之前的模型在交互层利用注意力机制融合段落和问题时都是利用句子的高层级别的语义信息而忽略了句子在低层次级别的语义信息如单词级别的词嵌入等。Huang等人\upcite{Fusionnet}提出FusionNet模型,将每一个单词在第一层到后面所有层的向量表示拼接成一个向量,原文中称为单词历史(history of word),因为它包含了一个单词所有层的语义编码。但是随着层数的增加维度会变得越来越大,为了解决维度问题同时不损失单词的历史信息,FusionNet提出全关注注意力机制的概念:即利用段落和问题的单词历史计算得到注意力权重,然后对问题的某一层语义向量加权求和。这种机制使得两个输入向量可以互相关注到对方的历史信息同时压缩维度,文中对注意力权重的计算方式如下:
\begin{equation}
\alpha_i=\text{ReLU}(Up_i)^TD\text{ReLU}(Uq_j)
\end{equation}
其中$p_i\in R^d$和$q_j\in R^d$分别代表段落第$i$个单词和问题第$j$个单词的单词历史,$U$和$D$是训练的参数。
%第三种方式是引入额外的记忆单元存储语义信息,目的是希望解决RNN中不能够长期依赖导致信息丢失的问题,典型
%	的如Weston等\upcite{MN}
%	提出的记忆网络(memory networks)。
%
Hu等人\upcite{RMR}认为在多层架构中,当前层的注意力计算并没有直接考虑到之前层计算得到的注意力信息,这可能导致两个不同但是相关的问题:(1)多层注意力分布集中在相同的文本上导致注意力冗余;(2)多层注意力未能集中在文本的重要部分造成注意力缺乏。针对这两个问题他们提出强化助记阅读器(Reinforced Mnemonic Reader,RMR)模型,利用重关注机制,通过直接利用之前层计算的注意力信息来微调当前层注意力分布的计算。

表7对比了本节介绍的经典的基于抽取式任务的MRC模型之间的差异。其中Q2C代表问题到段落注意力,C2Q代表段落到问题注意力,Bidirectional代表双向注意力,self-attention代表对段落做自注意力运算,one-dim代表一维注意力,two-dim代表二维注意力,one-hop代表单跳结构,multi-hop代表多跳结构。
\begin{table}[ht]
	%\text{基于注意力机制的模型对比}
	\centering
	\caption{基于注意力机制的模型对比 \\ Table 7 Comparison of models based on attention mechanism}
	%\vspace{10pt}
	%\resizebox{\linewidth}{!}{
	\begin{tabular}{c c c c}
		\toprule
		模型&注意力方向&注意力维度&推理模式 \\
		\midrule
		Attentive Reader\upcite{Hermann}&Q2C&one-dim&one-hop \\
		\midrule
		Impatient Reader\upcite{Hermann}&Q2C&two-dim&multi-hop \\
		\midrule
		Standford Reader\upcite{AR}&Q2C&one-dim&one-hop \\
		\midrule
		AS Reader\upcite{ASR}&Q2C&one-dim&one-hop \\
		\midrule
		IA Reader\upcite{IAReader}&Q2C&one-dim&multi-hop \\
		\midrule
		GA Reader\upcite{GAReader}&C2Q&two-dim&multi-hop \\
		\midrule
		Match-LSTM\upcite{MatchLSTM}&C2Q&two-dim&multi-hop \\
		\midrule
		DCN\upcite{DCN}&Bidirectional&two-dim&one-hop \\           
		\midrule
		BiDAF\upcite{BiDAF}&Bidirectional&two-dim&one-hop \\
		\midrule
		ReasoNet\upcite{Reasonet}&Bidirectional&two-dim&multi-hop\\
		\midrule
		R-Net\upcite{RNet}&C2Q+self-attention&two-dim&multi-hop \\
		\midrule
		RMR\upcite{RMR}&Q2C+self-attention&two-dim&multi-hop \\
		\midrule
		QANet\upcite{QANet}&Bidirectional&two-dim&one-hop\\
		\bottomrule
	\end{tabular}
	%}
\end{table}



\subsubsection{答案预测层}
这是整个模型架构的最后一层,用来输出预测的答案。MRC任务按照答案形式的不同大致分成四类,因此这一层的设计需要
考虑到答案形式。
%对于填空型任务,答案的输出是文章中的一个单词。对于多项选择任务,答案的输出是从多个候选答案中选择出正确的选项。
%对于片段选择型任务,答案的输出是文章中某段连续的文本。对于自由答案型任务,答案的输出不限固定的文本,而是根据词典中的单词生成文本。
%此外还有不可回答的问题,此时模型的输出还要考虑到问题是否可以回答。
%%答案预测层的设计细节见\ref{output}节。
%
%答案预测层的设计要依据答案的形式而设计,
下面介绍
各个模型在四类不同的MRC任务上的输出层设计。


1)填空式:这类任务答案的形式是预测问题中缺失的单词,而且缺失的答案来源于文章中。Hermann等人\cite{Hermann}
最早提出将问题的语义向量与问题感知的段落语义向量拼接成一个向量然后映射到整个词典中预测那个缺失的单词。
这种方法存在的一个问题就是不能够确保预测的单词一定是段落中的词汇,
这就使得模型的预测准确率受到影响。指针网络(Pointer networks\upcite{Ptr})模型由seq2seq模型演变而来,主要就是
为了解决输出源自于输入的问题,实现方式是利用计算的注意力的权重分布直接输出预测结果,而这种机制正适合
填空型任务以及片段选择型任务。
Kadlec等人\upcite{ASR}提出AS Reader模型正是受到指针网络的启发,对于计算得到的注意力权重分布,将其中相同单词的注意力
权值相加,最后输出具有最大权值的单词最为答案。
填空式任务模型的损失函数可以写
为$L(\theta)=-\displaystyle\frac{1}{N}\sum_{i=1}^{N}\log P_{y_i}$。
其中$\theta$为模型参数,$N$代表样本数目,$y_i$表示段落中第$i$个样本在段落中标准答案的位置。

2)多项选择式:这类任务是从多个候选答案选项中选择正确的选项。处理这种任务最简单的一种方式就是计算模型输出后的
段落语义信息和选项之间的相似程度,相似程度最高的作为预测的选项,从而将问题变化为句子之间的语义匹配问题。
Wang等人\cite{Co-matching}提出将问题、段落、选项一起放在模型中做交互计算输出一个向量作为输出层的输入,
输出层采用简单的输出维度是1的全连接层,输出的值代表模型对这个选项的打分值,其它的选项类似的处理,值最高的选项作为预测的答案。
最后对所有选项的打分值
做归一化作为模型的损失函数。
多项选择型任务模型的损失函数可以写为
$L(\theta)=-\displaystyle\frac{1}{N}\sum_{i=1}^{N}\sum_{j=1}^{m}\log P_{y_{j}^i}$。
其中$\theta$为模型参数,$N$代表样本数目,$m$代表选项个数,$y_{j}^i$表示第$i$个样本中第$j$个选项是正确答案。

3)抽取式:这类任务是从文章中提取出来一段连续的单词作为答案,虽然类似于填空型任务输出来源自输入的性质,但是
不像填空型任务仅仅只是预测一个单词。因此填空型
任务答案输出层的设计不能直接用来作为片段选择型任务答案预测层。
由于提取文本的长度不固定,使得这一任务更具有挑战性。
Wang等人\upcite{MatchLSTM}受到指针网络的启发提出了两种基于指针网络的输出模型,第一种是序列式模型,利用指针网络以一种序列式的形式生成答案的每一个位置,处理过程类似于seq2seq模型的解码过程,
这种模型下答案的每一个单词可能出现在文本段落的任何一个位置,
这是因为指针网络并没有要求从输入中选择的输出具有连续性。由于答案的长度不固定,因此在段落中设置一个特殊的
位置表示答案的终止点,当预测到这个位置时终止答案的生成。
第二种是边界式模型,不同于序列式模型那样序列的生成答案的每一个位置,由于要预测的答案是一段连续的文本,
因此可以利用指针网络仅仅预测答案的起始位置和终止位置。所预测答案的概率
是预测这两个位置概率的乘积,这种方式相比于
第一种更加的简单而且测试结果表明更加高效。


边界式模型的这种设计思想也被后来很多MRC模型采纳。尽管边界式模型简单有效,
%但是在文章中可能有些文本片段与标准答案相似,比如初始位置一样,
但是边界式模型有可能陷入局部极值的情况从而提取错误的文本片段。为了处理这个问题,Xiong等人\upcite{DCN}
提出一种动态迭代的指针网络作为解码端,利用上一次预测的答案的起始位置和终止位置以及解码端当前的状态来重新评估
下一次预测答案的起始位置和终止位置。多次迭代后选取所有迭代次数中概率最大的情形作为预测答案。
抽取式模型的损失函数可以写为
$L(\theta)=-\displaystyle\frac{1}{N}\sum_{i=1}^{N}\log P_{y_i^s}^S+\log P_{y_i^e}^E$。
其中$\theta$为模型参数,$N$代表样本数目,$y_i^s$表示第$i$个样本中标准答案的起始位置在文章中的位置,
$y_i^e$表示第$i$个样本中标准答案的终止位置在文章中的位置。
如果考虑到不可回答的问题,最简单的方式是额外在输出层加上一个输出维度是1的全连接层。
此时的损失函数可以写为
$L(\theta)=-\displaystyle\frac{1}{N}\sum_{i=1}^{N}(\log P_{y_i^s}^S+\log P_{y_i^e}^E)+\log P_{y_i^u}^U$。
其中$y_i^u$表示第$i$个样本中的问题是不可回答的问题。关于带有不可回答问题的阅读理解任务细节见\ref{unknown}节。

4)自由答案型:这类任务的答案形式已经不再是原文中某段文本,而是需要根据文章和问题生成符合语法规范的文本。
这类任务对答案生成模块的能力要求较高。处理生成任务典型的架构是seq2seq模型,See等人\upcite{PGNet}提出
一种指针生成网络模型(Pointer-Generator Network,PGNet),最早用在文本摘要领域,模型结合了seq2seq的
生成机制以及指针网络的拷贝机制,使得模型既能从词典中生成单词又能在原文中拷贝单词,实验结果表明该模型的效果优于传统的seq2seq模型。

表8对比了经典的基于抽取式任务的MRC模型在SQuAD\upcite{SQuAD1}数据集上的性能\footnote{统计数据源自Yu等人\upcite{QANet}}。


\begin{table}[ht]
	\centering
	\caption{模型在SQuAD\upcite{SQuAD1}数据集上的对比(acc代表准确率)}
	%\vspace{10pt}
	\begin{tabular}{l c}
		\toprule
		模型&EM/F1\\
		%\cmidrule(lr){2-2} \cmidrule(lr){3-3} \cmidrule(lr){4-4}
		%&EM/F1&EM/F1& acc \\
		\midrule
		Match-LSTM\upcite{MatchLSTM}& 64.7/73.7\\
		\midrule
		DCN\upcite{DCN}& 66.2/75.9\\
		\midrule
		BiDAF\upcite{BiDAF}&68.0/77.3\\
		\midrule
		ReasoNet\upcite{Reasonet}&70.6/79.4\\
		\midrule
		R-Net\upcite{RNet}&72.3/80.7 \\
		\midrule
		RMR\upcite{RMR}&73.2/81.8 \\
		\midrule
		QANet\upcite{QANet}& 76.2/84.6\\
		\bottomrule
	\end{tabular}
\end{table}
\subsection{复杂任务下的MRC模型}
以上介绍了经典的神经机器阅读理解模型并且详细的对比了各个模型在交互层注意力机制的差异。这些模型大多是
基于SQuAD\upcite{SQuAD1}数据集设计的。Weissenborn等人\upcite{fastqa}提出的FastQA模型,在编码层的输入中对段落的每一个单词额外的添加了两个特征(binary,weighted):binary特征表示原文中的词是否出现在问题中,weighted特征表示原文中的单词与问题的相似度。FastQA没有交互层复杂的注意力机制的设计,仅仅依靠这两个特征就在SQuAD数据集上取得了很好的效果。这一方面质疑那些复杂的注意力机制是否真的可以提升模型的效果,另一方面也说明SQuAD数据集难度不高,达不到测验模型推理和理解能力。
而且抽取式的问答要求答案是原文连续的文本片段,
显然不接近人类现实世界中的问答。
本节介绍复杂任务下的MRC模型,与之前的模型不同,在应对复杂的阅读理解任务下模型需要根据任务的特点来设计相应的结构。

% \begin{center}
%     \begin{figurehere}[ht]
%         \textbf{图1:CoQA\upcite{CoQA}数据集的一个样例}
%         \vspace{10pt}
%         \centering
%         \includegraphics[width=0.5\textwidth]{coqa.png}
%     \end{figurehere}
% \end{center}

\subsubsection{带有不可回答问题的阅读理解任务}\label{unknown}
之前的MRC数据集全都有一个共同的特点就是默认每一个问题都可以在给定的文本中找到答案,然而一段文本所包含的知识是限的,因此有下述两点是需要考虑的:(1)这段文本不能回答那些与文本表达内容无关的问题;(2)某些问题可能与文本内容类似但是问题含义与文本含义不同,这种问题仍然是不可回答的。
目前最流行的带有不可回答问题的数据集如SQuAD 2.0\upcite{SQuAD2},在SQuAD的基础上增加了五万多个不可回答的问题。一个样例如表9所示。

\begin{table}[ht]
	\centering
	\caption{SQuAD 2.0的一个样例 \\ Table 9 An example of SQuAD 2.0}
	%\vspace{5pt}
	\begin{tabular}{l p{13.2cm}<{\raggedright}}
		\toprule
		文章:&\tabincell{l}{Other legislation followed, including the Migratory Bird Conversation Act of 1929, \\ 
			a \textbf{1937 treaty} prohibiting the hunting
			of right and gray whales, and the \\ 
			\textbf{Bald Eagle Protection Act of
				1940}. These later laws had a low cost to society\\ —the species
			were relatively rare—and little opposition was raised.}\\
		%\cmidrule(l){2-3}
		\midrule
		问题:&\tabincell{l}{What was the name of the 1937 treaty} \\
		%\cmidrule(l){2-3}
		\midrule
		看似合理的答案:&Bald Eagle Protection Act \\
		\bottomrule
	\end{tabular}
\end{table}
从表9中可以看出题目问的是1937 treaty的名字,而Bald Eagle Protection Act指的是1940 treaty的名字,这对于模型来说是一个非常迷惑的答案。
对于这类任务模型必须区分出哪些问题是不可回答的,对于不可回答的问题模型不能再给出``貌似合理"的答案。在3.2节所介绍的模型里,很多模型在SQuAD数据集上表现很好然而在SQuAD 2.0数据集上效果显著下降,这说明很多模型只是基于浅层的语义匹配来寻找答案而不是真正的理解了文章的含义。

因此对于带有不可回答问题的阅读理解任务,模型要分为两个模块:(1)答案抽取模块;(2)判别不可回答问题模块。Clark等\upcite{Clark}尝试在原有的答案抽取模块的基础上额外添加一个专门用来预测不可回答情况的网络层,损失函数定义如下:

\begin{equation}
L_{joint}=-\log(\displaystyle\frac{(1-\delta)e^z+\delta e^{\alpha_a\beta_b}}{e^z+\sum_{i=1}^{l_p}\sum_{j=1}^{l_p}e^{\alpha_i\beta_j}})
\end{equation}
其中$z$表示模型预测该问题是不可回答问题的分数,如果问题是可以回答的则$\delta=1$,反之$\delta=0$。$\alpha$和$\beta$分别表示输出层预测的文章中每一个单词作为答案起始位置和终止位置的概率,$a$和$b$分别代表标准答案在文章中的起始位置和终止位置。

由公式可以看出预测的答案跨度分数$\alpha_a,\beta_b$和不可回答问题的分数$z$是共同归一化的。Hu等人\upcite{ReadVerify}认为两个分数共同归一化会出现冲突,如果模型过分信任预测的答案跨度分数那么就会在预测不可回答问题时产生较低的分数。此外之前的模型并没有验证答案抽取模块预测的答案跨度的合理性。
%所在的句子\footnote{原文中称为answer sentence}可以蕴含出这段跨度文本。
为了解决以上问题,他们提出Read+Verify架构。其中Read模块就是指答案抽取模块+判别不可回答问题模块,
Verify模块用来进一步验证是否答案抽取模块预测的答案跨度所在的句子(原文中称为answer sentence)就是标准答案所在的句子。
为了解决上面提到的冲突问题,在Read模块中额外增加了两个辅助损失函数:
\begin{gather}
L_{indep-span}=-\log(\displaystyle\frac{e^{\widetilde{\alpha}_{\widetilde{a}}\widetilde{\beta}_{\widetilde{b}}}}{\sum_{i=1}^{l_p}\sum_{j=1}^{l_p}\widetilde{\alpha}_{i}\widetilde{\beta}_{j}}) \\
L_{indep-unknown}=-(1-\delta)\log\sigma(z)-\delta\log(1-\delta(z))
\end{gather}
其中$L_{indep-span}$代表答案抽取模块的损失函数,而此时的答案抽取模块是独立的预测答案片段而不考虑问题是否可以回答,$\widetilde{\alpha}_{\widetilde{a}}$和$\widetilde{\beta}_{\widetilde{b}}$表示的就是答案抽取模块所预测出来的答案跨度。
$L_{indep-unknown}$代表判断问题不可回答的损失函数,同样它是独立于答案抽取模块的。$\sigma$代表sigmoid函数。
最后整个Read模块的损失函数定义为:
\begin{equation}
L_{Read}=L_{joint}+\gamma L_{indep-span}+\lambda L_{indep-unknown}
\end{equation}
$\gamma$和$\lambda$是两个超参数。实验表明去掉$L_{indep-unknown}$后模型在判断不可回答问题上的准确率显著下降,证明了上述提出的冲突确实存在。对于验证模块,他们采用三种结构。第一种将预测出来的答案片段连同问题以及answer sentence连接成一个句子送入预训练模型GPT\upcite{GPT}中预测不可回答的概率。第二种采用交互式结构,通过注意力机制计算它们之间的关联。第三种结构是前两个结构的结合,将前两个结构的输出张量拼接,实验证明这种混合结构使得模型效果更好。

%人在做阅读理解问题的时候通常会先带着问题大致的浏览一下这篇文章,对这篇文章的含义有一个大致的了解。之后再根据问题详细的阅读文章寻找答案。受到这种阅读形式的启发,Zhang等人\upcite{Retrospective}提出一种回顾式阅读器(Retrospective Reader,Retro-Reader)模型。整个模型由两个步骤构成:(1)第一步先简要的略读文章,建模文章与问题的大致关联给出初步的判断该问题是否可以回答。(2)第二步是精读模块,目的是验证可回答性并且给出最终判断。模型的编码器采用强大的预训练模型ALBERT\upcite{ALBERT},Retro-Reader在SQuAD 2.0数据集上显著优于其它模型。



\subsubsection{多段落型阅读理解任务}
多段落式阅读理解,即一个问题会对应着
多个相关的段落,也可以认为是开放领域(Open-domain)问答的一种形式。Open-domain问答目的是从广泛的领域资源(如维基百科,网页搜索等)寻找问题的答案而不限于仅仅在某段文本中,这更贴近于真实场景但同时具有相当大的难度。Chen等人\upcite{DrQA}提出利用检索+阅读(Retrieve+Read)的模式处理open-domain问答。具体的就是先利用检索模块(Document Retriever)从维基百科中获取5个与问题最相关的文章,然后利用阅读器(Document Reader)预测出答案所在的位置。其中Document retriever采用基于TF-IDF权重的词袋向量模型比较问题和文章的关联程度并且在此基础上用bigram哈希优化。

对于open-domain问答任务,检索模块要检索出与问题相关的文章,因此检索模块的性能极大地影响着模型整体的效果。如果简单的增加其检索文章的数量就可能导致有不相关的文章被检索出,仍然影响后续阅读模块。为了解决这个问题,Lee等人\upcite{Ranking}提出段落排序(Paragraph Ranker)机制,利用BiLSTM获得每一篇段落和问题的表示向量,然后计算两个向量的内积作为这篇段落与问题的相似度,目的是从多篇文章中的多个段落选出与问题最相关的几个段落。


目前典型的多段落型数据集如MS MARCO\upcite{MSmarco}、TriviaQA\upcite{TriviaQA},中文的有DuReader\upcite{DuReader}。
以MS MARCO数据集为例,数据集样例见表4. MS MARCO由微软亚洲研究院发布,问题和文章来源于
必应搜索,答案由人工生成,因此数据集接近真实应用场景而且答案不在局限于文章中
。每个问题对应10个由必应搜素引擎返回的文本段落,其中与问题答案相关的段落用$\text{is\_select=1}$标记为1。
Tan等人\upcite{SNet}提出S-Net模型,
先通过片段抽取模块提取出
一段文本作为答案的预测依据,然后利用生成模块生成答案。其中片段抽取模块采用多任务学习策略,
除了预测文本片段之外还添加一个段落排名任务,
将标记为$\text{is\_select=1}$的段落视为正例。答案生成模块采用seq2seq模型,
其中encoder端的输入是问题单词的向量表示以及将片段抽取模块的输出作为额外的特征
和文章单词的向量表示拼接。实验证明S-Net在MS MSRCO数据集上的效果要显著地优于
R-Net\upcite{RNet},ReasoNet\upcite{Reasonet}这些用来做片段抽取任务的模型。

多段落型阅读理解任务复杂的原因之一就是由于有多个段落,不同的段落都有可能会包含与问题语义相近的答案,但是有些答案并不是正确的。基于这个问题,Wang等人\upcite{VNet}提出一种模型使得来自不同段落的候选答案在基于它们所在的上下文内容里互相验证对方的正确性。将每一篇段落中预测出来的答案与其它段落预测的答案做交互验证。这样做的原因是因为相比于错误的答案,正确答案中的单词往往会在多个段落中重复出现,因此通过交互验证可以凸显出正确答案。最后模型在MS MARCO数据集上的效果优于S-Net。


\subsubsection{对话型问答任务}\label{cmrc}
%虽然本节介绍的对话型任务按照答案形式上划分仍然可以划分到那四种类型里面,但是由于对话型问答任务与其它任务在数据集构造方式以及任务形式上有较大不同,因此本节单独列出对话型问答任务。
%除了按照答案形式上划分还可以根据文章类型划分,如单段落型阅读理解还是多段落型阅读理解。
%
%
%根据问答形式划分,比如上面所介绍的所有数据集都属于单轮对话式问答,
%即文章所对应的多个问题之间没有联系,每一个问题都是互相独立的。这并不符合
%现实世界中人与人之间的对话交流,人们是通过多轮对话形式来交流的,每一轮的问题和答案都会影响后面的问答情况。
无论是单段落型阅读理解还是多段落型阅读理解任务,它们都属于单轮对话问答,即问答的形式只有一轮,后面的问题与前面的问题和答案无关,每一个问题都是互相独立的。
而在现实世界中人们是通过多轮对话形式来交流的,每一轮的问题和答案都会影响后面的问答情况。
所以对话型任务来讲,在回答当前轮的问题时不仅需要考虑文章还需要考虑前几轮的问题和答案。
具体可以表示为:给定$Q_i,D,Q_{i-1},\cdots,Q_{i-k}$以及$A_{i-1},\cdots,A_{i-k}$要求模型给出$A_{i}$。其中$Q_i,A_i$表示第$i$轮的问题和答案,$D$表示文章,$Q_{i-1},\cdots,Q_{i-k}$和$A_{i-1},\cdots,A_{i-k}$分别表示前$k$轮的问题和答案,建模概率:
\begin{equation}
P(A_i|D,Q_i,Q_{i-1},\cdots,Q_{i-k},A_{i-1},\cdots,A_{i-k})
\end{equation}

目前典型的对话型问答数据集有CoQA\upcite{CoQA}以及QuAC\upcite{QuAC}。
不同之处在于CoQA数据集的答案形式较为简单,类似于SQuAD\upcite{SQuAD1},但是包含有yes/no以及unknown问题,其中unknown代表不可回答问题,此外还有一定比例的问题是自由答案形式。而QuAC数据集的构造过程中提问者没有看过文章而仅仅了解文章的标题,由回答者根据文章的内容选择出文章的一段文本作为答案,这种数据集构造形式类似于用户在搜素引擎中输入问题查找答案,目的是减少问题和文本之间的依赖,使得模型尽量避免通过浅层的匹配方式获得答案。
对话型阅读理解数据集的一个样例见表10。

\begin{table}[ht]
	\centering
	\caption{CoQA\upcite{CoQA}数据集的一个样例 \\ Table 10 An example of CoQA}
	%\vspace{10pt}
	\resizebox{\textwidth}{!}{\begin{tabular}{l p{15.5cm}<{\raggedright}}
			\toprule
			\multirow{3}{*}{文章}&Jessica went to sit in her rocking chair. Today was her birthday and she was turning 80. Her granddaughter Annie was coming over in the afternoon and Jessica was very excited to see her. Her daughter Melanie and Melanie's husband Josh were coming as well.\\
			\cmidrule{2-2}
			\multirow{3}{*}{第一轮}&$Q_1$: Who had a birthday? \\
			&$A_1$: Jessica \\
			&$R_1$: Jessica went to sit in her rocking chair. Today was her birthday and she was turning 80. \\
			\cmidrule{2-2}
			\multirow{3}{*}{第二轮}&$Q_2$: How old would \textbf{she} be? \\
			&$A_2$: 80 \\
			&$R_2$: she was turning 80. \\
			\cmidrule{2-2}
			\multirow{3}{*}{第三轮}&$Q_3$: Did \textbf{she} plan to have any \textbf{visitors}? \\
			&$A_3$: Yes \\
			&$R_3$: Her granddaughter Annie was coming over \\
			\cmidrule{2-2}
			\multirow{3}{*}{第四轮}&$Q_4$: \textbf{How many?} \\
			&$A_4$: Three \\
			&$R_4$: Her granddaughter Annie was coming over in the afternoon and Jessica was very excited to see her. Her daughter Melanie and Melanie's husband Josh were coming as well. \\
			\cmidrule{2-2}
			\multirow{3}{*}{第五轮}&$Q_5$: \textbf{Who?} \\
			&$A_5$: Annie, Melanie and Josh \\
			&$R_5$: Her granddaughter Annie was coming over in the afternoon and Jessica was very excited to see her. Her daughter Melanie and Melanie's husband Josh were coming as well. \\
			\toprule  
	\end{tabular}}
\end{table}

% \begin{center}
% \begin{figurehere}
%     \textbf{图1:CoQA\upcite{CoQA}数据集的一个样例}
%     \vspace{10pt}
%     \centering
%     \includegraphics[width=0.5\textwidth]{coqa.png}
% \end{figurehere}
% \end{center}
其中每一个$Q_i$和$A_i$代表问题和对应的答案,每一个$R_i$表示给出这个答案的依据,用来训练模型。
测试集中是没有答案依据的。
%因此对话型问答任务可以描述为给定文章$P$和
%历史的对话信息$Q_1,A_1,Q_2,A_2,\cdots,Q_{k-1},A_{k-1}$,任务的目的是
%给出第$k$轮的问题$Q_k$的答案$A_k$。
从图中可以清楚地看到$Q_2$和$Q_3$中的she指代的是$Q_1$的答案$A_1$,而$Q_4$的How many?以及$Q_5$的Who?所问的
是$Q_3$中的visitors。显然仅仅靠一轮的问题是无法回答的,对话历史信息在对话型问答任务中尤为重要。

Reddy等人\upcite{CoQA}采用三种模型在CoQA数据集上进行实验。
第一种是传统的seq2seq模型,decoder用来生成答案。第二种是指针生成网络(Pointer-Generator Network\upcite{PGNet},PGNet),既可以从词典中生成答案又可以从原文中拷贝单词,很好的解决了OOV(Out Of Vocabulary)问题。
第三种是DrQA+PGNet模型,其中DrQA\upcite{DrQA}是一个片段抽取模块。整个模型的思想是
先利用片段提取模块从文章中提取中与问题最相关的一段文本,然后利用答案生成模块在这个被抽取出来的文本上
生成答案,实验结果表明这种结合模型的效果是优于前两个模型的。为了能够利用历史的对话信息,做法是将前几轮的问题与答案结合到
文章当中作为上下文来回答当前轮的问题。
%\begin{center}
%    \begin{figurehere}
%        \textbf{图2:CoQA\upcite{CoQA}数据集的一个样例}
%        \vspace{10pt}
%        \centering
%        \includegraphics[width=0.5\textwidth]{coqa.png}
%    \end{figurehere}
%\end{center}

Choi等人\upcite{QuAC}利用BiDAF++\upcite{Clark}模型在QuAC数据集上进行实验,为了利用历史的对话信息,在文章中设置一个标记向量用来标记文章中的单词是否出现在历史答案中,在问题向量的基础上添加问题的轮次,这是另一种处理历史对话信息的方式。
%Yatskar等人\upcite{CompareSQC}采用同样的模型以及同样的历史信息处理方式在CoQA数据集上进行实验,由于CoQA数据集包含有yes/no问题,因此在输出层额外设计预测yes/no的情况。

Huang等人\upcite{FlowQA}认为上述的方法只是简单的添加之前轮的问题和答案,而忽略了在回答之前轮问题时模型对整篇文章的推理过程状态,他们提出一种带有流机制的模型FlowQA,目的是将模型处理每一轮的问答过程下的对文章的语义理解状态流向下一轮的问答过程。FlowQA模型整体上利用双向循环神经网络编码文章,利用单向循环神经网络编码对话历史,对比之前的模型,FlowQA能够集成更加深层次的对话历史状态。

%虽然预训练模型BERT\upcite{BERT}在自然语言理解任务上展现出其强大的性能,但是其数据输入形式只能是两个句子的拼接因此并不适合直接处理对话型任务。Qu等人\upcite{HAE}提出一种简单而有效的模型,仅仅需要在BERT模型的输入端为每一个单词添加两个额外的向量用来表明这个单词有没有在历史答案中出现过,文中称这两个向量为(History Answer Embedding,HAE)。实验表明BERT+HAE模型较之前的模型可以处理更多的历史对话信息。
%Zhu等人\upcite{SDNet}
%提出的SDNet模型,以基于特征的方式迁移BERT作为编码器,同时将之前轮次的问题和答案拼接到当前轮的问题上构成一个新问题。模型采用自注意力机制获得历史对话信息之间的交互语义,具体的计算方式采用FusionNet\upcite{Fusionnet}模型提出的融合方法。
%Ohsugi等人\upcite{simpleqa}以基于微调的方式迁移BERT模型。
%将历史对话信息每一轮的问题与答案分别与文章连接送入
%BERT,将每一个BERT的输出连接作为输出层的输入。





值得注意的是尽管CoQA数据集有部分答案是自由答案形式的,但是上面的模型大多是利用片段提取式的做法在CoQA数据集上实验,主要原因在于
生成式模型的效果往往不如提取式模型的效果好,因为生成式模型对答案生成模块要求较高。
因此如何提高模型的答案生成效果是值得进一步研究的方向。
%由于预训练模型UNILM\upcite{UNILM}改进了BERT的训练任务,
%增加了自回归语言模型以及seq2seq语言模型使得其
%在生成式任务上的效果很好,在CoQA数据集上远远的超过于Reddy等\upcite{CoQA}提出的基准模型。

%下面介绍神经机器阅读理解模型中基于注意力机制的模型和基于预训练的模型。
%\input{embedding_layer.tex}



%\subsection{基于注意力机制的模型}
交互层是整个网络模型中关键的一层,前面的编码层输出的是问题和文章中每个单词的上下文语义编码,每个单词
关注了自己所在句子的上下文单词,但是却并没有关注对应的句子。而我们在做阅读理解问题时,通常是带着
问题去文章中找答案,我们要知道文章中每一个单词和问题之间的相关度。因此交互层的目的就是让文章的语义信息与问题的
语义信息融合,以此达到对文章更深层次的理解,而交互层中最常用的方法就是注意力机制。

注意力机制简单的来说就是为序列的每一个位置生成一个权重值,这个值代表当前位置的单词对预测结果
的重要性。文献\cite{neural machine translation by jointly learning to align and translate}最早将
注意力机制应用在机器翻译领域,获得了极大的反响。这种注意力的思想就是在解码端每次生成一个单词
时要对编码端的句子做关注,也就是注意力的计算,具体的就是利用解码端的单词与编码端序列计算出一个权值分布,这个权值表示的
是编码端每一个单词对解码端单词的重要程度,然后利用权值分布对编码端序列加权求和得到一个固定长度的向量,这个向量就是注意力的计算结果
,并且作为下一个单词输入的一部分,这属于一种动态式的计算方式。
而这种
在将注意力计算结果送入到循环神经网络中并且对于循环神经网络的
隐藏状态同时参与注意力计算的网络也叫基于注意力的循环神经网络。
以下将首先介绍MRC任务中注意力机制之间的差异,然后分析各个模型将注意力机制应用在MRC任务上以及它们之间的区别与联系。

\subsubsection{MRC中的注意力}
% 单方向注意力的运算机制类比于人类做阅读问题的过程: 带着问题到文章中找答案,从文章的角度
% 对问题进行总结,获得文章对问题的注意力。
% 也就是先看问题,然后对于文本段落中各个部分生成一个注意力权重,代表这个部分与问题的相关程度,
% 利用注意力权重对文本的各个部分加权求和就可以得到文本段落与问题之间的相关性信息。
不同于机器翻译任务,只能将encoder端的单词语义信息融入到decoder端。
在机器阅读理解问题上,文章和问题都可以作为encoder或者decoder。这种情况下做注意力运算有两个方向,即从问题到文章,从文章到问题这两个方向。从问题到文章的注意力是指利用注意力权值
对文章的语义表示加权求和,将文章的语义信息融合到问题上。从文章到问题注意力是指将问题的语义信息融合到文章中,即文章的每一个单词都会
关注问题。
以问题到文章的注意力计算过程为例,$P=[p_1,p_2,\cdots,p_n]$代表文章的语义信息,其中的每一个
$p_i$代表文章中单词的语义向量表示,$Q$代表整个问题的语义信息。问题到文章的注意力就是利用问题
的语义信息和文章中每一个单词的向量表示计算它们之间的相关性,最后得到一组注意力权重$\alpha=[\alpha_1,\alpha_2,\cdots,\alpha_n]$
表示文章与问题之间的相关程度,利用$\alpha$对文章加权求和就可以提取出来文章中与问题最相关的单词,
这些单词对于回答问题是至关重要的。
具体计算过程如下:
\begin{gather}
    \alpha_i=\text{softmax}(s(p_i,Q)) \notag \\
    C=\sum_{i=1}^{n}\alpha_ip_i
\end{gather}
其中$s(p_i,Q)$就是计算问题
的语义信息和文章中每一个单词的向量表示它们之间的相关性的一个函数,常用的计算方式有点积、双线性项、
加法模型,分别见如下公式:
\begin{gather}
    s(p_i,Q)=p_i^TQ \\
    s(p_i,Q)=p_i^TWQ \\
    s(p_i,Q)=v^T\tanh(Wp_i+UQ)
\end{gather}
上面的式子是将问题压缩成一个固定维度的向量,得到的注意力权重也是一维的,因此也成为一维注意力,与其相对应的是二维注意力,即对于问题中的
每一个单词都会和文章做注意力计算,得到的注意力权重是二维向量。

如果按照注意力的计算形式上区分,
又可以分为one-hop和multi-hop形式。one-hop,也叫“单跳结构”是指仅仅通过一次计算得到注意力权值然后加权求和得到注意力结果,这也是一种
静态的计算形式。与之对应的是multi-hop,也叫“多跳结构”。one-hop形式下仅仅只做一次交互计算,而注意力机制虽然可以提取相关的重要信息,但是
它仍然是基于浅层语义信息的相似度计算。在机器阅读理解任务中,对于复杂的问题通常是不能在一个句子中找出答案,需要多步推理才能寻找答案,如下图中
的例子
% \begin{figure*}
%     \includegraphics[]{multihop.png}
% \end{figure*}
%当前时刻计算的注意力结果要保留到下一时刻,即每一时刻都要计算注意力,是一种动态的计算形式,典型的代表是Bahdanau注意力\cite{neural machine translation by jointly learning to align and translate}。
我们可以看到想要得到最终的答案需要进行三次推理过程,在每一个推理过程中都会变换注意力关注的对象。
实现多步推理这种机制通常有三种方式:
1)基于之前时间步所计算得到的关注问题的文章语义信息计算下一时间步的文档和问题交互,如Impatient Reader\upcite{Teaching Machines to Read and Comprehend}。
2)利用RNNs这种基于上一时刻隐藏状态更新下一时刻隐藏状态的循环特性来达到多步推理,如Match-LSTM\upcite{Machine comprehension using match-lstm and answer poi
nter}。
3)引入额外的记忆单元存储语义信息,目的是希望解决RNN中不能够长期依赖导致信息丢失的问题,典型的如文献\cite{memory networks}
提出的记忆网络(memory networks)。

此外还有自注意力机制,自注意力机制是文本的向量表示和自身做交互,即文本中所有单词之间都会计算注意力,与单词的
位置顺序无关,这样对于两个距离较远的单词仍然可以交互,
循环神经网络这种序列式的计算机制非常适合处理文本句子这种序列式数据,但是也正因如此,网络的训练
是非常耗时的同时存在梯度消失的问题使得模型不能捕获长距离单词之间的关联信息。
自注意力机制(self attention)主要的思想就是撇弃这种序列式的计算方式,对于一个句子
中的两个单词不考虑单词之间顺序的关系,直接计算它们之间的相关度,例如计算两个单词向量表示的内积。
Self attention可以捕获句子中长距离依赖的特征关系,而且这种计算方式是并行的,极大地加快了模型的训练速度。
解决了循环神经网络固有的序列式传递信息导致后面的单词与前面的单词
之间达不到有效的信息传递问题。

下表详细的列举了本文介绍的所有模型的注意力机制。

\begin{center}
    %\textbf{}
    \resizebox{\linewidth}{!}{
        \begin{tabular}{c c c c}
            \toprule
            模型&注意力方向&注意力维度&推理模式 \\
            \midrule
            Attentive Reader\upcite{Teaching Machines to Read and Comprehend}&问题到文章&一维注意力&单跳结构 \\
            \midrule
            Impatient Reader\upcite{Teaching Machines to Read and Comprehend}&问题到文章&二维注意力&多跳结构 \\
            \midrule
            Standford Reader\upcite{AR}&问题到文章&一维注意力&单跳结构 \\
            \midrule
            AS Reader\upcite{ASR}&问题到文章&一维注意力&单跳结构 \\
            \midrule
            IA Reader\upcite{IAReader}&问题到文章&一维注意力&多跳结构 \\
            \midrule
            GA Reader\upcite{GAReader}&文章到问题&一维注意力&多跳结构 \\
            \bottomrule
        \end{tabular}
        }
\end{center}




\subsubsection{相关模型}

%\subsubsection{Attentive Reader and Impatient Reader}
文献\cite{Teaching Machines to Read and Comprehend}最早利用神经网络模型并且融入注意力机制做MRC任务。
文中提出两种不同的单向注意力机制Attentive Reader
和Impatient Reader,均是计算问题到文章的注意力。
Attentive Reader是将问题表示为一个固定长度的向量然后与文章中每一个单词做注意力计算,然后利用注意力权重对文章
中的单词的向量表示加权求和得到一个固定长度的
向量即为注意力运算后的结果,然后与问题联合预测答案,可见这种注意力的计算方式仅仅只计算一次,因此也叫one-hop。
这种方式类似于人在阅读时带着问题去文章中找答案。
Impatient Reader计算注意力的机制类似于文献\cite{neural machine translation by jointly learning to align and translate}
的方式,也属于一种multi-hop的计算。并不是将
问题表示为一个固定长度的向量,而是对于问题中的每一个单词都要和整个文章做注意力计算,而且计算的结果
要和下一个单词以及文章共同做注意力计算,最后一个单词的注意力结果作为整个Impatient Reader计算注意力过程的结果。
这种方式类似于人在阅读过程中不断的在问题和文章之间做关注。
文献\cite{AR}在Attentive Reader的基础上利用双线性项(公式8)取代原有的利用tanh函数的加法模型(公式9)并且直接将
对文章加权求和后得到的向量作为预测答案的输入而不是联合问题$Q$的语义信息,实验证明这种简化反而提高了模型的准确度。

文献\cite{memory network}提出的记忆网络并不能端到端的训练。为了处理这个问题,文献\cite{MemN2N}提出一种端到端的记忆网络(MemN2N)。
利用记忆槽存储文档中每一个句子的嵌入矩阵,记忆槽的状态以及问题的语义信息会随着与文档的多次交互不断更新。

文献\cite{IAReader}提出Iterative Attention Reader(IA Reader)模型,利用BiGRU\upcite{BiGRU}存储每一次
迭代计算得到的问题和文章的交互信息。在每一时间步上,首先利用上一次的BiGRU的状态与问题做一维注意力匹配提取出问题的语义信息,
然后再结合上一次的BiGRU的状态与文章再做一维注意力匹配从而提取出文章的语义信息。将问题与文章的语义信息
通过各自的门控单元作为BiGRU当前时刻的输入,其中门控单元采用前馈神经网络用来解决当前时间步下问题和文章的语义信息提取不充分的问题。

文献\cite{GAReader}提出Gated Attention Reader(GA Reader)模型,类似于IA Reader\upcite{IAReader}模型同样采用
BiGRU作为编码模块实现多跳结构。在每一步的推理过程中,首先通过BiGRU得到问题的语义信息然后对文章的
每一个单词做注意力的计算,同时采用点乘计算的门控机制建模
当前时间步的推理过程下关注了文章第$i$个单词的问题语义信息与文章第$i$个单词的语义信息影响,从而更新文章的语义表示。这种处理过程
类比于带着问题反复的阅读文章,每一次都加深对文章的语义理解。

文献\cite{Reasonet}提出一种动态决定推理次数的模型ReasonNet,不像之前的模型如GA Reader\upcite{GAReader}
,IA Reader\upcite{IAReader}等在整个推理过程中有着固定的推理次数。这种固定推理次数的缺点就是不考虑问题的复杂性,
对于复杂的问题往往需要模型多次的推理,因此不同题目难度需要不同的推理次数,应当让模型学会什么时候终止推理。为了达到这一目的,
ReasoNet模型利用一个终止门产生二元值输出来动态的决定是否继续推理。Reasonet模型大致分为外部记忆单元模块、内部控制器模块、终止门模块以及答案输出模块。
具体的,将文章和问题通过Bi-GRU编码后的语义表示作为
外部的记忆单元$M$,利用内部控制器(采用GRU)当前时刻的状态$s_t$与$M$做二维注意力匹配,得到注意力结果输入到内部控制器中
更新内部控制器的状态。终止门模块以当前时刻内部注意力的状态作为输入来判断是否需要继续推理。由于产生了二元离散输出值,
使得模型用梯度下降法训练,因此模型引入强化学习机制训练。

%\subsubsection{Match-LSTM}
文献\cite{MatchLSTM}提出一种Match-LSTM的交互机制,利用RNNs的循环机制达到多跳结构。
与之前模型如Impatinent Reader\upcite{Teaching Machines to Read and Comprehend}不同,
虽然Match-LSTM也是多跳结构,并且也是一维注意力,但是
Match-LSTM计算的方向是文章到问题的注意力,也就是对问题的文本表示加权求和,将问题的语义信息融入到Match-LSTM中。
具体的就是对于文章中当前时刻单词的上下文表示,以及Match-LSTM前一时刻的隐藏状态与问题的语义编码做注意力的计算,
计算方式和Bahdanau\upcite{neural machine translation by jointly learning to align and translate}一样
采用激活函数
为tanh的全连接层以及输出维度是1的全连接层,最后利用softmax将结果以概率形式归一化
得到注意力权重。具体计算过程如下:
\begin{gather}
    s_t=v^T\tanh(W^QH^Q+W^Ph_t^P+W_rh_{t-1}^r) \notag \\
    \alpha_{t}=\text{softmax}(s_t)
    %c_t=\sum_{i=1}^{m}a_i^tu_i^Q
\end{gather}
其中$H^Q$是问题通过编码层的输出,$h_t^P$是文章的第$t$个单词通过编码层的输出,$h_{t-1}^r$是Match-LSTM上一时刻
的隐藏状态。
$\alpha_{t}$是
文章的第$t$个单词与问题的每一个单词之间的注意力权重。
计算得到的注意力权重对问题的语义编码加权求和,此时得到的是关注问题的向量表示,然后与文章当前时刻单词的上下文表示拼接作为
Match-LSTM当前时刻的输入。
\begin{gather}
    z_t=[h_t^p;\alpha_tH^q] \notag \\
    h_{t}^r=\text{LSTM}(z_t,h_{t-1}^r)
\end{gather}
此外为了使得文章从后向前的对问题做关注,将文章序列翻转再次按照上述方式计算,最后将两个方向的计算结果
拼接作为交互层的输出。这种multi-hop的计算方式比较耗时。

%Match-LSTM仅仅计算文章到问题的注意力,并未考虑问题到文章的注意力。







%\subsubsection{DCN}
文献\cite{Dynamic coattention networks for question answering}提出一种Dynamic Co-attention Network(DCN)模型,
在交互层中采用协同注意力\cite{VQACo}机制,
这种机制首先在视觉问答领域提出,目的是将图片到问题,以及问题到图片两个方向的注意力以某种形式融合。
定义$D=[d_1,d_2,\cdots,d_m,d_{\phi}] \in R^{l\times(m+1)}$,$Q^{'}=[q_1,q_2,\cdots,
q_n,q_{\phi}] \in R^{l\times(n+1)}$。
,其中$d_{\phi}$和$p_{\phi}$是一个监哨向量\upcite{sentinel vector},在做注意力交互机制计算时,对于那些
某些在文章和问题中不相关的单词,模型将这些单词映射为这个监哨向量,使模型并不会
关注到这些不相关的单词。为了使得使得问题编码空间和段落文本编码空间有可变化的余地,在问题编码向量$Q^{'}$上再利用一个非线性层把$Q^{'}$转换为
$Q=\tanh(W^{(Q)}Q^{'}+b^{(Q)})\in R^{l\times(n+1)}$,$Q$作为问题的最终表示。
协同注意力网络同步的计算文章对问题的注意力以及问题对文章的注意力。具体的,首先计算关联矩阵$L$然后利用关联矩阵
得到文章端的注意力权重$A^Q$以及问题端的注意力权重$A^D$:
\begin{gather}
    L=D^{T}Q\in R^{(m+1)\times(n+1)} \notag \\
    A^Q=\text{softmax}(L)\in R^{(m+1)\times(n+1)} \notag \\
    A^D=\text{softmax}(L^T)\in R^{(n+1)\times(m+1)} \notag 
\end{gather}
$A^Q$的每一列表示的是问题的一个单词对文章所有单词的相关度,它的编码空间是在文章端,同理
$A^D$的编码空间是在问题端。因此接下来利用$A^Q$计算
问题对文章的注意力$C^Q=DA^Q\in R^{l\times(n+1)}$。对于文章对问题的注意力,定义
$$
C^D=[Q;C^Q]A^D\in R^{2l\times(m+1)}
$$
其中$C^QA^D$的含义是将问题端的编码信息转换到文章端,
其中$[;]$表示向量在行的维度上连接,$C^D$是问题和文章的协同依赖的注意力矩阵,它的每一个值都是既关注到了文章的语义信息,
也关注到了问题的语义信息。最后融合
$C^D$和文章的向量表示$D$作为交互层的输出。可以看出DCN模型中注意力仅仅只计算一次,而且两个方向的注意力可以并行计算,
这属于one-hot形式的注意力机制,速度要显著快于Match-LSTM。

%\subsubsection{BiDAF}
文献\cite{Bidirectional attention flow for machine comprehension}提出Bidirectionl Attention Flow(BiDAF)模型。
对于之前的模型如Attentive Reader\upcite{Teaching Machines to Read and Comprehend},
Impatient Reader\upcite{Teaching Machines to Read and Comprehend},Match-LSTM\upcite{MatchLSTM}等,
这些模型计算注意力的方式都是将注意力结果问文章的语义信息融合作为交互层的输出,
而BiDAF额外将注意力结果也作为交互层输出的一部分,这在一定程度上避免了过早的对问题语义信息概括而导致
信息的损失,使得交互层计算得到的注意力结果仍然参与后面网络层的计算。
BiDAF的计算方式如下:
\begin{gather}
    S=W^T[C;Q;C\circ Q]\in R^{m\times n} \notag \\
    \widehat{Q}=\text{softmax}(S)Q \in R^{m\times d}\notag \\
    \alpha=\text{softmax}(\max_{col}(S)) \in R^{m}\notag \\
    \widehat{c}=\sum_{i=1}^{m}\alpha_tC_{:i}\in R^{d}
\end{gather}
其中$C,Q$分别表示文章和问题的语义向量表示,$m,n$分别表示文章和问题的长度,
$S$表示文章和问题之间的相似度矩阵,$\circ$表示点乘运算。$\widehat{Q}$表示的就是文
章对问题的注意力,即代表着问题中哪个单词与文章最相关。
$\alpha$的意义是将关于问题最相关的文章中的单词提取出来,这些单词是作为答案的关键因素。
$\widehat{c}$就是将文章
中与问题最相关的单词的语义向量加权求和,将其重复$m$次得到问题对文章的注意力$\widehat{C}$。
最后以如下形式作为BiDAF交互层的输出:
\begin{gather}
    [C;\widehat{Q};C\circ \widehat{Q};C\circ \widehat{C}]
\end{gather}
此外实验结果表明去掉文章到问题的注意力后模型效果显著下降。

%\subsubsection{RNet}
文献\cite{RNet}提出一种带有门控机制的注意力循环神经网络以及自注意力机制联合的交互层设计模型,
也叫RNet。RNet在交互层的设计分为两部分。
第一部分是带有门控机制的注意力循环神经网络,整体计算
方式类似于Match-LSTM\upcite{MatchLSTM},不过采用的是RNN不是LSTM而且额外加入了门控机制使得模型可以有选择的输出
语义信息。具体的,在公式(2)中的$z_t$上添加一个门控单元:
\begin{gather}
    g_t=\text{sigmoid}(W_gz_t)\notag \\
    z_t^{*}=g_t\odot z_t
\end{gather}
其中$\odot$表示元素之间的点乘。
由于$z_t=[h_t^p;\alpha_tH^q]$,$h_t^p$表示的是文章的第$t$个单词的语义表示,$\alpha_tH^q$表示的是
对问题语义表示的融合,因此
通过添加门控单元使得模型可以有选择的决定哪部分作为重要的语义信息输出。这种机制类似于人在阅读过程中要
忽略文章中那些与问题无关的信息,凸显出重要的信息才能更加准确的找到答案。
第二部分是利用自注意机制对文章的语义信息再次交互建模。基于注意力机制的循环神经网络的输出对关注了问题的
文章语义表示与原始文章语义表示建模后的输出,而这种计算机制的问题之一是
在它输出的整个句子的语义信息中前面的单词没有关注到后面的单词
而两个距离较远的单词之间交互信息由于链式求导等原因会变得很弱。因此
通过自注意机制可以使得文章中每一个单词关注到其余所有的单词,使得模型对文章达到更深层次的理解。

% 文献\cite{FusionNet}提出一种融合网络(FusionNet),所谓融合就是指将问题与文章的语义信息交互,然后将文章的语义信息融合到问题中以及将
% 问题的语义信息融合到文章中,一个好的融合机制很大程度上影响模型的性能。一般认为对于网络中较低的层所表达的是语法层面的特征,


%\subsubsection{QANet}
文献\cite{QANet}提出一种网络模型QANet,不像之前的那些模型几乎都是用RNN的变体来做编码器,
QANet提出一种新颖的编码结构,利用卷积结合transformer\upcite{Transformer}中的multi-head注意力结构
,实验结果表明这种架构不仅加快训练速度同时在SQuAD数据集上模型性能优于那些利用RNN作为编码器的模型。
QANet交互层的计算方式类似于BiDAF\upcite{Bidirectional attention flow for machine comprehension}
以及DCN\upcite{Dynamic coattention networks for question answering}。
具体的:
\begin{gather}
    S=W[Q,C,Q\odot C]\in R^{m\times n}\notag \\
    \bar{S}=\text{softmax(S)}\in R^{m\times n}\notag \\
    \widehat{S}=\text{softmax($S^T$)}\in R^{n\times m}\notag \\
    A=\bar{S}Q^T\in R^{m\times d} \notag \\
    B=\bar{S}\widehat{S}C^T
\end{gather}
其中$Q\in R^{d\times n},C\in R^{d\times m}$分别问题和文章通过
编码层的语义表示,$m,n,d$分别表示文章长度和问题长度以及输出维度。
$\odot$表示点乘,$W$是一个训练参数,
$A$表示就是文章到问题的注意力,$B$表示的是问题到文章的协同注意力。
其中通过$\bar{S}$将问题的
语义编码空间转换到文章中,类似于DCN\upcite{Dynamic coattention networks for question answering}
的计算方式。交互层的输出采用与BiDAF\upcite{Bidirectional attention flow for machine comprehension}
一样的输出形式,见公式(4)。
%One possible interaction for the operation C^QA^D is tha mapping of question encodings into space
%of passage encodings















%
%
\subsection{基于预训练模型的MRC模型}
% 使用预训练模型可以解决缺少标注数据的问题,因为预训练的模型已经在大量的无标签数据集上学习到了一些重要的特征等,因此应用到
% 具体任务时不需要大量的标注数据,也不需要长时间的训练即可达到很好的效果。
% 语言模型是NLP领域一个非常基础但同时也是非常
% 重要的一项任务,语言模型的目标就是根据一个单词的上下文预测出这个单词。
% 按照预测方式的不同可分为自回归语言模型和自编码语言模型,主要区别在于训练方式上是基于自回归形式还是自编码形式。
预训练模型近年来NLP领域获得了极大的关注度,基于预训练模型的方法在NLP几乎所有的任务上都要优于之前的模型。
预训练方式源自于迁移学习的概念: 首先在其它相关任务上预训练模型,使得模型已经学习到一些知识,然后在目标任务上做进一步优化,
实现模型所学知识的迁移。
对于NLP领域来讲,预训练过程就是在大量的文本数据上学习到通用的语言表示。在应用到下游任务时,预训练所学习到的知识提供了一个很好的
初始化点,从而加快模型的收敛并且提高模型的性能。此外预训练也对模型起到正则化的作用,使得模型避免在数据不充分的数据集上过拟合。

前面介绍的ELMo\upcite{ELMo}就是一个预训练模型,但是预训练的层次比较浅对模型的提升效果也是有限的。
目前流行的几个预训练模型如,
GPT\upcite{GPT},BERT\upcite{BERT}以及基于BERT改进的预训练模型RoBERTa\upcite{RoBERTa},UNILM\upcite{UNILM},ALBERT\upcite{ALBERT}等,从MRC模型结构的角度看这些预训练模型相当于将上述模型通用结构的编码层和交互层融合在一起,在编码的同时进行段落与问题的交互,这些预训练模型刷新了MRC领域多个任务的最佳性能。
%全都是基于语言模型做预训练,因此也叫预训练语言模型。
%预训练的模型根据迁移方式的不同可以分为两种方式基于特征的方式(Feature-based)
%和基于微调的方式(Fine-tuning)。
%基于特征的方式是指对于预训练好的模型,应用到具体任务时固定模型的参数,将具体任务的数据通过预训练的模型得到数据的表示
%特征然后输入到根据具体任务设计的具体模型中。基于微调的方式是广泛采用的一种预训练模型的使用方式,具体的做法是在预训练模型的基础上加上少量的网络层,然后利用具体任务的
%标注数据训练整个网络模型。
%下面将主要概述NLP领域一些流行的预训练模型以及它们在MRC任务上的应用和效果对比。

\subsubsection{Transformer}\label{transformer}
鉴于目前几乎所有的预训练模型都采用transformer\upcite{Transformer}结构或者其变体作为模型的特征提取器,因此本节首先介绍transformer结构。
Transformer是由Vaswani等人提出了一种用于机器翻译的序列到序列(seq2seq)结构。
Encoder端由六个相同的层堆叠而成,每一层有两个子层,第一个子层采用多头(multi-head)自注意力机制,第二个子层采用前馈神经网络(Feed-Forward Network,FFN)构成。
之所以用自注意力机制是因为它既可以捕获句子中每一个单词的全局依赖关系而不受距离影响又可以并行计算。
对比公式(1),自注意力机制下$Q=K=V$,transformer中采用的计算方式如下:
\begin{equation}
\text{Attention}(Q,K,V)=\text{softmax}(\frac{QK^T}{\sqrt{d_k}})V
\end{equation}
其中$\sqrt{d_k}$代表张量维度。此外transformer采用的是多头(multi-head)自注意力机制,将$Q,K,V$三个张量线性映射成多份,每一份之间做注意力的运算最后拼接。
\begin{gather}
	\text{head}_i=\text{Attention}(QW_i^Q,KW_i^K,VW_i^V) \notag \\
	\text{Multi-head}(Q,K,V)=\text{concat}(\text{head}_1,\text{head}_2,\cdots,\text{head}_h)W^o
\end{gather}
其中$h$代表头的数目,是一个超参数,$W_i^Q,W_i^K,W_i^V,W^o$都是训练参数。

采用multi-head的目的是让模型联合关注序列中不同位置单词的不同表示子空间的信息,可以类比于卷积神经网络中利用多个卷积核做特征提取,目的同样是使得不同的卷积核关注的不同的特征。
此外
每一个子层都利用层正则化(layer normalization\upcite{layerNormal})和
残差连接(residual connection\upcite{RL})机制。
Decoder端与Encoder端类似,区别在于每一层额外添加了encoder-decoder注意力。

%Transformer的整体架构
%通过利用自注意力(self-attention)机制取代RNN那种序列式的计算方式,
%对于一个句子中的两个单词不考虑单词之间顺序的关系,直接计算它们之间的相关度,例如计算两个单词向量表示的内积。
%自注意力机制可以捕获句子中长距离依赖的特征关系,
%解决了循环神经网络固有的序列式传递信息导致后面的单词与前面的单词
%之间达不到有效的信息传递问题。
%通过自注意力机制不仅可以做到
%单词之间的全局交互同时其并行计算使得模型训练时间大幅减少。
Transformer的encoder端和decoder端都可以做特征提取器如BERT\upcite{BERT}用encoder端特征提取,GPT\upcite{GPT}用decoder端特征提取,实验证明在大规模数据集上transformer的特征提取能力要强于
基于RNN变体的编码器,目前几乎所有的NLP预训练模型都是利用transformer作为特征提取器。


\subsubsection{预训练模型}\label{pretrain}
%\subsubsection{相关预训练模型}
%ELMo\upcite{ELMo}是在2018年提出的一种预训练语言模型。
%传统的词嵌入模型如Word2Vec\upcite{word2vec},GloVe\upcite{GloVe}属于
%静态的词向量,训练好模型后一个单词的表示
%向量就是固定的,没有考虑上下文的信息,因此无法解决多义词问题。
%ELMo提出一个三层网络的模型
%,第一层就是词嵌入层用来提取单词特征,随后是两层BiLSTM网络分别提取
%单词的词性特征和语义特征。前向LSTM的目标是根据前$k-1$个词预测第$k$个词,从而计算出
%一个句子的概率,如公式(10)。反向LSTM的目标是根据最后的单词直到第$k+1$个单词预测第$k$个单词,
%具体如公式(11)。
%\begin{gather}
%    p(t_1,t_2,\cdots,t_N)=\prod_{k=1}^{N}p(t_k|t_1,t_2,\cdots,t_{k-1})\\
%    p(t_1,t_2,\cdots,t_N)=\prod_{k=1}^{N}p(t_k|t_{k+1},t_{k+2},\cdots,t_{N})
%\end{gather}
%最后的目标函数是最大化联合的前向和后向最大似然:
%\begin{equation}
%    \begin{split}
%    L(\Theta)&=\sum_{k=1}^{N}(\log p(t_1,\cdots,t_{k-1};\Theta)) \\
%        &+\log p(t_{k+1},\cdots,t_N;\Theta)
%    \end{split}
%\end{equation}
%在做阅读理解任务时,将文章和问题输入到模型中,每一层都会得到句子的语义表示,
%然后将每一层的特征加权求和作为ELMo的输出,此时得到的每一个单词的
%向量表示都是考虑了上下文的,将其作为下游模型嵌入层的输入。
%利用ELMo+BiDAF\upcite{BiDAF}
%结构超过之前单模型8.3个百分点。可以看出ELMo属于自回归语言模型,模型的迁移方式是基于特征的方式。




%\subsubsection{GPT}
OpenAI\upcite{GPT}提出一种生成式预训练模型(Generative Pre-Trained,GPT),使用多层的transformer\upcite{Transformer}的解码端作为特征提取器。模型采用两阶段的训练方式,先利用大规模无监督语料库训练前向语言模型,
然后在下游任务的少量监督数据集上微调模型,因此
GPT也是一种半监督学习方式,
%采用单向语言模型作为训练的任务,
预训练阶段的目标函数就是标准的单向语言模型的目标函数,见公式(1)。
%\begin{equation}
%    L(\Theta)=\sum_{k=1}^{N}(\log p(t_1,\cdots,t_{k-1};\Theta))
%\end{equation}
%GPT与ELMo的共同点是它们都属于自回归语言模型,
%而最大的不同之处在于迁移的设计上并不是像ELMo那种基于特征方式的迁移。GPT有统一的输入数据表示形式,
%而且在输入层并不需要继续设计复杂的网络模型而仅仅只需要简单的结构,在训练时整个网络模型的参数共同训练。
%这种基于微调的迁移方式被后续的其它预训练模型广泛采用。
%利用GPT在RACE\upcite{RACE}数据集上微调后达到的效果比之前最好的模型提高了5.7个百分点。
GPT是NLP领域首先提出的一种基于微调(fine-tune)的通用式网络结构,不仅仅在MRC领域,在很多其它的NLP领域都取得了很大的
进步。
%\end{multicols}
% \begin{center}
%     \textbf{表2 预训练语言模型对比}\\
%     \vspace{10pt}
%     \begin{tabular}{c c c c}
%         \toprule
%         模型&任务&模型结构&介绍 \\
%         \midrule
%         ELMo&单向LM&LSTM&拼接两个单向语言模型的语义信息,
                
%         基于特征形式迁移 \\
%         \midrule
%         GPT&单向LM&Transformer&首次采用预训练+微调形式
        
%         的两阶段任务,特征提取器采用transformer \\
%         \midrule
%         BERT&双向LM+NSP&Transformer&利用掩码语言模型和预测下一个句子共同作为训练任务 \\
%         \midrule
%         UNILM&单向LM+双向LM+Seq2SeqLM&Transformer&同时训练三种语言模型,采用掩码机制解决不同语言模型的约束问题 \\
%         \midrule
%         ALBERT&双向LM+SOP&Transformer&对比BERT采用矩阵分解和共享参数减少模型的参数量同时用句子顺序预测取代下一个句子预测 \\
%         \bottomrule
%     \end{tabular}
% \end{center}


%\subsubsection{BERT}
GPT属于自回归语言模型,自回归语言模型的缺点就是由于自回归的性质使得它不能同时利用一个单词的
上下文信息预测这个单词
Devlin等人\cite{BERT}提出BERT预训练模型,
与GPT最本质的不同在于预训练方式上采用的降噪自编码方式,
随机掩盖
掉一些单词,在输出层获得掩盖位置的概率分布,让模型根据掩盖位置的上下文预测这个单词,
这种机制也叫掩码语言模型(Masked Language Model,MLM)或双向语言模型。
%自回归语言模型的缺点就是由于自回归的性质使得它不能同时利用一个单词的
%上下文信息预测这个单词,
%ELMo虽然利用双向LSTM来预测单词但是这也只是两个单向的语言
%模型的拼接并不能当做双向语言模型。
%BERT在整个预训练的流程上采用GPT的方式,即预训练然后微调。
%但是BERT采用降噪自编码(DAE)的方式训练,具体的就是在输入数据中加入噪声,也就是随机掩盖
%掉一些单词,让模型根据掩盖掉单词的上下文预测这个单词,这种训练方式也叫掩码语言模型(MLM)。
%对比ELMo和GPT的目标函数,BERT的掩码语言模型的
MLM的目标函数为:
\begin{equation}
    \begin{split}
        L(\Theta)=\sum_{i=1}^{N}\log P(t_k|t_1,t_2,\cdots,t_{k-1},t_{k+1},\cdots,t_{N})
    \end{split}
\end{equation}
除了MLM任务外,还利用下一个句子预测(Next Sentence Prediction,NSP)任务使得模型
在诸如文本蕴含、问答这类需要判断两个句子关系的下游任务表现更好。
BERT的预训练过程实质上是
一个多任务学习的过程,通过MLM和NSP两个任务
提高了预训练模型的语义表达能力,其中MLM任务用来学习句子中词与词之间的语义关联而NSP任务用来学习两个句子之间的逻辑关系。
BERT在SQuAD\upcite{SQuAD1}数据集上的效果超过了人类的水平,在其它的NLP任务上也都有提升。
%\subsubsection{UNILM}

BERT开启了NLP领域预训练模型的时代,此后很多更加强大的预训练模型相继提出。这些预训练模型从结构上主要分为两大类:基于BERT的改进模型和XLNet。基于BERT的改进模型主要是针对BERT预训练阶段的两个任务MLM和NSP做改进,如
Liu等人\upcite{RoBERTa}提出RoBERTa模型,使用动态掩码替换BERT的静态掩码同时去除NSP任务;Li等人\upcite{UNILM}提出UNILM模型扩展了BERT预训练的任务,由于双向语言模型的性质使得BERT在生成任务上效果不好,
UNILM同时训练单向语
言模型(包括从左到右和从右到左)、双向语言模型以及Seq2Seq语言模型,使用掩码机制来解决不同的语言模型
约束问题;Lan等人\upcite{ALBERT}提出ALBERT模型,利用句子顺序预测(Sentence Order Prediction,SOP)任务改进了BERT的NSP任务;除了在预训练阶段改进训练任务外,Liu等人\upcite{MTDNN}提出MT-DNN模型在微调阶段引入了多任务学习机制,使用多个任务来微调模型参数使得模型具有更好的泛化型。

BERT模型以及基于BERT结构改进的模型在预训练阶段都采用降噪自编码的思想,所带来的问题就是预训练过程与微调过程不匹配,预训练使用的掩码机制在下游任务微调时不会被使用,导致两个过程存在数据分布的偏差。自回归语言模型虽然不存在这个问题,但是缺点是不能同时利用上下文信息。Yang等人\upcite{XLNet}提出的XLNet模型是一种可以同时获得上下文信息的自回归语言模型,
通过使用排列语言模型,引入双向自注意力流和循环机制兼顾了自回归和自编码语言模型的优点。

%
%BERT这种降噪自编码的方式虽然可以达到双向的利用上下文信息,但是由于其在预训练过程中对输入数据加入掩码而在微调时又不会加入掩码,导致
%了预训练过程与微调过程不匹配,存在一定的数据分布偏差,此外BERT对于屏蔽词的预测是独立的。基于上述问题,文献\cite{XLNet}提出一种新的
%预训练模型XLNet,它是一种可以获得双向的上下文信息的自回归语言模型,克服了传统的自回归语言模型和自编码语言模型各自的问题。XLNet采用的是
%排列语言模型(Permutation Language Model,PLM)
%,自回归模型中利用文本的前向或者后向序列的最大似然来建模,而PLM排列这个文本所有可能
%的序列顺序,仍然利用自回归语言模型的目标函数,但是综合所有可能的排列后每一个单词都可以获得双向的上下文信息。此外XLNet
%借鉴transformer-XL\upcite{Transformer-XL}中的片段循环机制引入循环机制可以捕获更长的句子依赖关系。

%文献\cite{RoBERTa}提出一种基于BERT改进其训练方式的预训练模型RoBERTa,其改进的方式包括:使用动态掩码替换静态掩码、去除NSP任务、使用更大的batch、
%更多的训练语料以及更长的训练时间。
%其中动态掩码是指对于输入数据中随机掩盖的单词并不是固定的,因为BERT的掩码机制是静态的,即对于每一个输入序列一旦选定了其中的
%的某个单词将其屏蔽,那么之后的整个训练过程该单词始终被掩盖。RoBERTa提出将输入数据复制10份,每一份都是随机掩盖部分单词,这样同一个输入序列
%就会有10种不同的掩码方式,从而达到动态掩码的目的。





%
%
%UNILM\cite{UNILM}扩展了BERT预训练的任务,由于双向语言模型的性质使得BERT在生成任务上效果不好,
%UNILM同时训练单向语
%言模型(包括从左到右和从右到左)、双向语言模型以及Seq2Seq语言模型,使用掩码机制来解决不同的语言模型
%约束问题。虽然是三个不同的语言模型作为训练任务但是共享同一个网络结构,也就是利用三个任务来联合的优化
%模型的参数。这种多任务学习的方式缓解了模型在某一个单一任务上容易出现过拟合的问题,
%使得预训练后的模型不仅在原有的自然语言理解任务上效果进一步提升,
%同时在自然语言生成任务上也达到了很好的效果。
%在微调阶段,对于自然语言理解任务(如文本分类、抽取式问答等)同BERT的微调方式一样,对于自然语言
%生成任务(如自动化摘要、生成式问答等)采用与预训练阶段Seq2Seq语言模型类似的方式在目标序列中
%随机的掩盖掉一些单词从而让模型根据源序列生成目标序列的单词达到生成任务的目的。
%UNILM不仅在抽取式QA数据集(如SQuAD)上超过BERT,在生成式QA数据集(如CoQA)上的表现远超过最初的基准模型。




%ALBERT\upcite{ALBERT}改进了BERT的NSP任务,BERT的NSP任务包含了两个子任务,来源于同一篇文章的两个连续的句子判别为正例,来源于不同文章的句子判别为负例,
%这使得模型不能集中于判断两个句子之间的顺序反而更加关注句子所表达的主题是否一致。因此ALBERT中用
%SOP(sentence-order prediction)取代NSP任务,SOP是指句子顺序预测,两个句子都是
%来源于同一篇文章的连续的句子,调换顺序后便是负例,这使得模型集中于预测句子之间的顺序关系,
%实验表明SOP任务使得模型的效果提升一个百分点。
%ALBERT的改进机制使得模型的预训练后的效果更好,在RACE\upcite{RACE},SQuAD 2.0\upcite{SQuAD2}等
%机器阅读理解数据集上的准确率超过BERT以及其它的预训练模型。
表11从预训练任务、模型采用的特征提取器等详细对比了本文介绍的所有预训练模型。表12对比了几个预训练模型在两个常用的MRC数据集上的表现。
% \subsection{小结}
% 传统的静态的预训练词向量虽然可以给模型性能带来一定程度的提升,但是这种提升非常有限。主要原因是
% 其静态的性质无法解决多义词的
% 问题而且所学习到的语义信息也是浅层的,仍然需要模型在标注数据集上从头学习。
% ELMo的出现使得静态词向量的问题得以解决,并且这种利用语言模型作为训练任务的方式也被后续的其它
% 预训练模型采用。GPT首次提出预训练过程结合微调过程的这种两阶段方式,使得不需要设计复杂的网络模型即可在
% 多个NLP任务上达到显著地提升。BERT是NLP预训练史上重要的里程碑模型,后续的其它更加优秀的
% 预训练模型如ALBERT、UNILM也都是在BERT的基础上进行进一步优化如:引入生成任务、改进训练方式等。

\begin{table}[ht]
	\caption{预训练模型对比\\ Table 11 Comparison of pre-trained model}
	\centering
	%\vspace{10pt}
	%表格超出页边距用resizebox
	\resizebox{\textwidth}{!}{
		\begin{tabular}{c c c c}
			\toprule
			模型&任务&模型结构&介绍 \\
			\midrule
			ELMo\upcite{ELMo}&\tabincell{c}{前向LM \\ 反向LM}&LSTM&\tabincell{c}{拼接两个单向语言模型的语义信息,\\ 基于特征形式迁移} \\
			\midrule
			GPT\upcite{GPT}&前向LM&Transformer&首次采用预训练+微调形式 \\
			\midrule
			BERT\upcite{BERT}&MLM+NSP&Transformer&利用掩码语言模型(MLM)和下一句预测(NSP)共同作为训练任务 \\
			\midrule
			MT-DNN\upcite{MTDNN}&MLM+NSP&Transformer&预训练过程与BERT一致,微调阶段采用多任务学习 \\
			\midrule
			XLNet\upcite{XLNet}&PLM&Transformer-XL&\tabincell{c}{采用排列语言模型(PLM),双向自注意力流\\模型以自回归方式训练但是基于上下文预测} \\
			\midrule
			RoBERTa\upcite{RoBERTa}&MLM&Transformer&采用动态掩码机制,去除NSP任务 \\
			\midrule
			UNILM\upcite{UNILM}&\tabincell{c}{前向LM+反向LM\\ +MLM\\ +Seq2SeqLM}&Transformer&\tabincell{c}{同时训练多种语言模型,\\ 采用掩码机制解决不同语言模型的约束问题} \\
			\midrule
			ALBERT\upcite{ALBERT}&MLM+SOP&Transformer&\tabincell{c}{对比BERT采用矩阵分解和共享参数减少模型的参数量,\\ 同时用句子顺序预测任务(SOP)取代下一个句子预测任务(NSP)} \\
			\bottomrule
		\end{tabular}
	}
\end{table}

 \begin{table}[ht]
     \centering
     \caption{预训练模型对比}
     \begin{tabular}{l c c}
         \toprule
         \multirow{2}{*}{模型}& SQuAD 2.0\upcite{SQuAD2} & RACE\upcite{RACE}\\
         \cmidrule(lr){2-2} \cmidrule(lr){3-3} 
         &EM/F1& Acc \\
         \midrule
         $\text{GPT}_{v1}$\upcite{GPT}&-&59.0\\
         \midrule
         $\text{BERT}_{large}$\upcite{BERT}& 80.0/83.1&72.0\\
         \midrule
         XLNet\upcite{XLNet}&86.4/89.1 &81.8\\
         \midrule
         RoBERTa\upcite{RoBERTa}&86.8/89.8 &83.2\\
         \midrule
         ALBERT\upcite{ALBERT}&88.1/90.9 &86.5\\
         \bottomrule
     \end{tabular}
 \end{table}


%\subsubsection{的MRC模型}
%在预训练模型出现后不仅仅在MRC任务上,在其它的NLP任务上模型的结构都发生了较大的变化,
%在模型的基础上利用具体任务的数据微调模型即可达到很好的效果。
%%但是预训练模型毕竟是一个通用式的模型,
%%它给了模型很好的初始化参数,不过要想达到更好的效果仍然要根据具体任务的形式来设计模型。
%下面列举几个MRC数据集上目前效果较好的基于预训练模型的MRC模型。
%
%人在做阅读理解问题的时候通常会先带着问题大致的浏览一下这篇文章,对这篇文章的含义有一个大致的了解。之后再根据问题详细的阅读文章寻找答案。受到这种阅读形式的启发,Zhang等人\upcite{Retrospective}提出一种回顾式阅读器(Retrospective Reader,Retro-Reader)模型。整个模型由两个步骤构成:(1)第一步先简要的略读文章,建模文章与问题的大致关联给出初步的判断该问题是否可以回答。(2)第二步是精读模块,目的是验证可回答性并且给出最终判断。模型的编码器采用强大的预训练模型ALBERT。Retro-Reader在SQuAD 2.0\upcite{SQuAD2}数据集上显著优于其它模型。
%
%%如何训练出一个性能更加强大的预训练模型成为近年来NLP领域的研究热点。
%Zhang等人\upcite{DCMN}
%利用BERT\upcite{BERT}和XLNet\upcite{XLNet}作为编码器同时采用文献\cite{Co-matching}提出的co-matching方法提出了DCMN模型,在RACE\upcite{RACE}
%数据集上达到了很高的准确率。
%而后在DCMN的基础上引入选项交互模块和段落选择模块提出DCMN+\upcite{DCMN+}模型,模型的性能进一步提升。
%
%BERT以及基于BERT改进的预训练模型其数据输入形式只能是两个句子的拼接因此并不适合直接处理CoQA这种对话型阅读理解数据集。
%%Qu等人\upcite{HAE}提出一种简单而有效的模型,仅仅需要在BERT模型的输入端为每一个单词添加两个额外的向量用来表明这个单词有没有在历史答案中出现过,文中称这两个向量为(History Answer Embedding,HAE)。实验表明BERT+HAE模型较之前的模型可以处理更多的历史对话信息。
%Zhu等人\upcite{SDNet}
%提出SDNet模型,以基于特征的方式迁移BERT作为编码器,同时将之前轮次的问题和答案拼接到当前轮的问题上构成一个新问题。模型采用自注意力机制获得历史对话信息之间的交互语义,具体的计算方式采用FusionNet\upcite{Fusionnet}模型提出的融合方法。
%Ohsugi等人\upcite{simpleqa}以基于微调的方式迁移BERT模型。
%将历史对话信息每一轮的问题与答案分别与文章连接送入
%BERT,将每一个BERT的输出连接作为输出层的输入。两个模型在CoQA\upcite{CoQA}和QuAC\upcite{QuAC}两个对话型数据集上的效果均超过前面的BiDAF++\upcite{Clark},FlowQA\upcite{FlowQA}等利用复杂交互机制的模型。
%由于预训练模型UNILM\upcite{UNILM}改进了BERT的训练任务,
%增加了自回归语言模型以及seq2seq语言模型使得其
%在生成式任务上的效果很好,在CoQA数据集上远远的超过于Reddy等\upcite{CoQA}提出的基准模型。




\subsubsection{迁移预训练模型}
预训练模型具有强大的文本表征能力,但是在应用到机器阅读理解任务上还需要根据具体的数据集特点设计不同的微调网络结构。以BERT系列的预训练模型为例,预训练阶段和微调阶段除了输出层外共享相同的体系结构,在做抽取式阅读理解任务时,输出层最简单的设计方式是仅利用两个向量与段落的向量表示计算答案在段落中起始位置和终止位置的概率。
对于其它形式的阅读理解任务,往往根据任务的特点在输入层和输出层设计相应的结构,相关的工作可以参照\upcite{HAE,simpleqa,Retrospective}。
另一种迁移预训练模型的方式是基于特征形式的迁移,将具体任务的数据送进预训练模型,预训练模型的特征表示作为下游模型的输入,训练时固定预训练模型的参数。


%预训练模型根据迁移方式的不同可以分为两种方式:(1)基于特征的方式(Feature-based)
%(2)基于微调的方式(Fine-tuning)。
%基于特征的方式是指将具体任务的数据送进预训练模型,将得到的文本表示
%特征输入下游模型结构中。
%
%基于微调的方式是广泛采用的一种预训练模型的使用方式,具体的做法是在预训练模型的基础上加上少量的网络层,然后利用具体任务的
%标注数据训练整个网络模型。



虽然设计一个强大的预训练模型具有更好的泛化型和应用价值,但是预训练模型所消耗的计算资源是巨大的,
因此如何利用预训练模型结合具体任务改进模型的结构是至关重要的。
% \begin{center}
%     \begin{tabular}{l c}
%         \toprule
%         单模型&EM/F1 \\
%         \midrule
%         MatchLSTM with Pointer Network& 64.7/73.7 \\
%         \midrule
%         Dynamic Coattention Network& 66.2/75.9 \\
%         \midrule
%         BiDAF&68.0/77.3 \\
%         \midrule
%         R-Net&72.3/80.7 \\
%         \midrule
%         QANet(data augmentation x3)& 76.2/84.6\\
%         \midrule
%         ELMo+BiDAF&78.6/85.8\\
%         \midrule
%         $\text{BERT}_{large}$(+TriviaQA)& 85.1/91.8 \\
%         \bottomrule
%     \end{tabular}
% \end{center}
% \begin{table*}[!hp]
%     \centering
%     \begin{tabular}{l c c c c}
%         \toprule
%         &SQuAD 1.1(F1)&SQuAD 2.0(F1)&RACE(Acc)&CoQA \\
%         \midrule
%         ELMo+BiDAF&85.8&-&-&- \\
%         GPT1&
%         \bottomrule
%     \end{tabular}
% \end{table*}

% \begin{table}
%     \begin{tabular}{l c}
%         \toprule
%         训练任务&目标函数 \\
%         \midrule
%         前向语言模型&$P(t_1,t_2,\cdots,t_N)=\prod_{k=1}^{N} P(t_k|t_1,t_2,\cdots,t_{k-1})$ \\
%         \midrule
%         反向语言模型&$P(t_1,t_2,\cdots,t_N)=\prod_{k=1}^{N} P(t_k|t_{k+1},t_{k+2},\cdots,t_N)$ \\
%         \midrule
%         掩码语言模型&$P(t_1,t_2,\cdots,t_N)=\prod_{k=1}^{N} P(t_k|t_1,t_2,\cdots,t_{k-1},t_{k+1},t_{k+2},\cdots,t_N)$ \\
%         \bottomrule
%     \end{tabular}
% \end{table}

% \subsubsection{迁移预训练模型}
% 训练好一个模型后应当考虑如何迁移这个预训练模型到具体的任务上。
% 按照迁移形式的不同可以分为(1)基于特征的迁移方式和(2)基于微调的迁移方式。
% % 基于特征的迁移方式是指将数据输入到预训练模型中然后提取出模型对输入数据的特征表示
% % ,将这组特征作为下游任务模型的输入,训练时预训练模型的参数是固定的。这种基于特征的迁移形式
% % 相当于把预训练模型当做语义编码器,上层仍需要设计具体的模型。

% % 基于微调的迁移方式是指将数据输入到预训练模型中,然后将预训练模型的输出作为
% ELMo\upcite{ELMo}是典型的基于特征的迁移方式,

% GPT\upcite{GPT}是典型的

%% %\section{Model}

\subsection{词嵌入层}
如何有效的表示文本是NLP领域最基础也是最重要的问题之一,早期的表示文本的方式有one-hot形式或词袋模型。
one-hot也称独热编码,将单词表示为一个只有0和1两个值的向量,大小和语料库中词典的大小相同,每一个单词都有自己的编号,
对应于自己编号的位置是1,其余位置都是0。显然当词典很大时,这种表示方式存在着维度灾难,数据稀疏以及单词之间没有任何的语义信息关联
等问题。

词袋模型(Bag-of-words)通过构建一个向量表示一段文本,向量的长度同样是词典的大小。
模型忽略掉文本中单词的顺序,将文本简单的看做是若干词汇的集合,然后统计一段文本中单词出现的次数作为该单词在向量中的表示。
虽然这种方法没有严重的数据稀疏问题,但是由于没有考虑单词的语序,语义等重要信息因此也不能够很好的表示文本。

\subsubsection{分布式表示}
分布式表示是将单词用一个低维度的稠密向量表示,即将单词嵌入到一个低维稠密空间中,因此这种表示方式也叫词嵌入。
Mikolov\upcite{word2vec}提出的word2vec模型是一种被广泛使用的词嵌入模型,训练方式包括CBOW模型以及Skip-gram模型,
两个模型都是借鉴N-gram模型的思想,利用固定的窗口对单词周围的上下文建模,例如CBOW利用一个单词的上下文预测该单词,
而Skip-gram利用单词预测这个单词的上下文。Mikolov利用这种


\textbf{\zihao{5} MatchLSTM and Pointer Network}
文献\upcite{Machine comprehension using match-lstm and answer pointer}是首个将神经网络应用
在SQuAD数据集上的模型,并且超过Rajpurkar et al.\upcite{SQuAD1}使用逻辑回归和手工构造特征的方法。
模型由两部分组成,match-LSTM\upcite{MatchLSTM}和pointer networks\upcite{Pointer Networks}。

Match-LSTM最初是用来做文本蕴含的模型,在文本蕴含任务中,给定一个前提和一个假设,判断假设是否蕴含在
前提中。对于假设句子的每一个单词,match-LSTM利用注意力机制计算这个单词和前提的相似度,
计算的结果与这个单词原本的向量表示拼接然后送入一个LSTM网络中,这个LSTM也就是match-LSTM。
在MRC任务中,可以把文章看做假设,问题看做前提。按照上述过程计算得到带有关注问题的文章编码表示。




为了在相反的方向上获得带有关注问题的编码表示,将文章序列翻转然后通过match-LSTM,计算结果再翻转回来。
最后将正向和反向的结果拼接作为指针网络层的输入。

Pointer networks是从输入序列中选择部分位置的单词作为输出,这正适合抽取式问答这种任务,因为输出结果是输入中的一部分。
Pointer networks利用注意力机制获得输入序列每一个单词的概率分布,与经典的Bahdanau\upcite{neural machine translation by jointly learning to align and translate}
注意力思想不同,并不是利用计算的注意力权重分布对输入加权求和,而是利用概率分布直接作为预测的依据,选择概率最大的那个单词作为输出。
根据输出形式的不同分为两种模型。

第一种是序列式模型,利用指针网络以一种序列式的形式生成答案的每一个位置,处理过程类似于seq2seq模型的解码过程,
这种模型下答案的每一个单词可能出现在文本段落的任何一个位置,
这是因为指针网络并没有要求从输入中选择的输出具有连续性。由于答案的长度不固定,因此在段落中设置一个特殊的
位置表示答案的终止点,当预测到这个位置时终止答案的生成。

第二种是边界式模型,不同于序列式模型那样序列的生成答案的每一个位置,由于要预测的答案是一段连续的单词的组合,
因此可以利用指针网络仅仅预测答案的起始位置和终止位置。所预测答案的概率
是预测这两个位置概率的乘积,这种方式相比于
第一种更加的简单而且测试结果表明更加高效。




边界式模型的这种设计思想也被后来很多MRC模型采纳



\textbf{\kaishu\zihao{5} Dynamic Coattention Networks}
文献\upcite{Dynamic coattention networks for question answering}提出一种动态协同注意力网络模型,由一个计算段落文本和问题
之间相关性的协同注意力编码器以及可以动态的评估所预测答案的起始位置和终止位置概率的动态指针解码器,
这种动态的迭代评估预测位置的概率避免了预测结果陷入局部最优的情形。

$(x_1^Q,x_2^Q,\cdots,x_n^Q)$表示问题中每个单词的词向量,$x_1^D,x_2^D,\cdots,x_m^D$表示文本段落中
每一个单词的词向量。使用LSTM\upcite{lstm}来作为编码器,
对于段落文本的第$t$个单词的上下文表示为$d_t=LSTM_{enc}(d_{t-1},x_t^D)$,
对于问题的第$t$个单词的上下文表示为$q_t=LSTM_{enc}(q_{t-1}+x_t^Q)$,
定义$D=[d_1,d_2,\cdots,d_m,d_{\phi}] \in R^{l\times(m+1)},Q^{'}=[q_1,q_2,\cdots,q_n,q_{\phi}] \in R^{l\times(n+1)}$。
,其中$d_{\phi}$和$p_{\phi}$\upcite{sentinel vector}是一个监哨向量,在做注意力交互机制计算时,对于那些
仅在段落文本中出现的一些不相关的单词或者仅在问题中出现的一些不相关的单词,模型将这些单词映射为这个监哨向量,使模型并不会
关注到这些不相关的单词。为了使得使得问题编码空间和段落文本编码空间有可变化的余地,在问题编码向量$Q^{'}$上再利用一个非线性层把$Q^{'}$转换为
$Q=\tanh(W^(Q)Q^{'}+b^{(Q)})\in R^{l\times(n+1)}$,$Q$作为问题的最终表示。

协同注意力网络同步的计算段落对问题的注意力以及问题对段落的注意力。具体的,首先计算关联矩阵$L$然后利用关联矩阵
得到问题对段落文本的注意力权重以及段落文本对问题的注意力权重:
\begin{gather}
    L=D^{T}Q\in R^{(m+1)\times(n+1)} \notag \\
    A^Q=softmax(L)\in R^{(m+1)\times(n+1)} \notag \\
    A^D=softmax(L^T)\in R^{(n+1)\times(m+1)} \notag 
\end{gather}

$A^Q$的每一列表示的是问题的一个单词对文章所有单词的相关度,因此接下来利用$A^Q$计算
文章-问题的注意力$C^Q=DA^Q\in R^{l\times(n+1)}$。对于问题-文章的注意力,定义
$$
C^D=[Q;C^Q]A^D\in R^{2l\times(m+1)}
$$
其中$[a;b]$表示向量在行的维度上连接,$C^D$是问题和文章的协同依赖的注意力矩阵,最后融合
$C^D$和文章的向量表示$D$送入一个BiLSTM层。
$$
u_t=Bi-LSTM(u_{t-1},u_{t+1},[d_t;c_t^D])\in R^{2l}
$$
定义$U=[u_1,u_2,\cdots,u_m]\in R^{2l\times m}$作为动态指针解码段的输入,从而预测
答案的位置。

对于之前的模型如\upcite{Machine comprehension using match-lstm and answer pointer}在预测答案的位置时可能会陷入
局部极值点的情况,也就是说答案的终止位置是根据起始位置预测的,如果起始位置预测错误那么
最后的预测结果肯定不是最优的。因此动态指针解码端通过迭代过程预测答案的位置可以使得从初始位置不正确
的答案中重新改正位置。为了使得模型能够从这种局部最优情况跳出来,提出了一种Highway Maxout Network(HMN),迭代的预测
答案的起始位置和终止位置,利用LSTM来存储上一次迭代预测结果的状态,在迭代的过程中,
解码端根据当前估计的答案的始末位置在$U$中的向量表示,通过HMN网络后估计新的答案的始末位置,当
两次估计的答案位置一致或者达到最大迭代次数则停止迭代。HMN是由两部分构成,
一部分是Highway Network\upcite{Highway Network},通过门控机制的思想使得输入一部分做非线性变换一部分直接与非线性变换后的结果相连接,
这样使得反向传播时原来输入的一部分的梯度可以直接流向那部分,缓解了梯度消失问题,从而可以训练较深的网络。
另一部分是Maxout Network\upcite{Maxout Network},它是一层激活函数可以学习的网络,而问答任务是由
多种不同的问题类型和文章主题构成,因此在评估答案位置时应该动态的调整网络,而Maxout网络通过可学习的激活函数
可以达到这一点。

\begin{figure}
    \centering
    \includegraphics[width=1.0\linewidth]{dcn_dynamic_decoder.png}
    \caption{Dynamic decoder}
\end{figure}

$s_{i-1}$和$e_{i-1}$指的是第$i-1$次迭代所预测答案的起始位置和终止位置,
$u_{s_{i-1}},u_{e_{i-1}}$指的是第$i-1$次迭代所预测答案的起始位置和终止位置的那个单词在
协同注意力层输出矩阵$U$的向量表示。
\begin{gather}
s_i=\underset{t}{argmax}(\alpha_1,\cdots,\alpha_m)\notag \\
e_i=\underset{t}{argmax}(\beta_1,\cdots,\beta_m) \notag
\end{gather}
其中$\alpha_t$和$\beta_t$代表文章中第$t$个单词作为答案起始位置和终止位置的分数
$\alpha_t$和$\beta_t$的计算方式为:
\begin{gather}
    \alpha_t=\text{HMN}_{start}(u_t,h_i,u_{s_{i-1}},u_{e_{i-1}}) \notag \\
    h_i=\text{LSTM}_{dec}(h_{i-1},u_{s_{i-1}},u_{e_{i-1}}) \notag 
\end{gather}

$h_{i-1}$是第$i-1$次迭代时解码端的LSTM的隐藏状态,$u_t$是文章中第$t$个单词在$U$中的向量表示。
利用$h_{i-1},u_{s_{i-1}},u_{e_{i-1}}$计算出第$i$次迭代时LSTM的隐藏状态$h_i$,然后利用$h_i$以及
文章中每一个单词在$U$中的向量表示通过三层HMN网络计算出$\alpha$,$\alpha$的每一个值表示的就是文章中
每一个单词在第$i$次迭代作为答案起始位置的分数。

注意$\beta_t=\text{HMN}_{end}(u_t,h_i,u_{s_i},u_{e_{i-1}})$





\textbf{\kaishu \zihao{5} BiDirectional Attention Flow}
文献\upcite{Bidirectional attention flow for machine comprehension}提出一种双向注意力流模型,简称BiDAF。
首先在不同的语义粒度上表示文本,包括字符层面、单词层面、以及上下文层面。然后将问题或者段落的上下文表示送入到注意力流层。
注意力流层不同于之前流行的注意力机制
\begin{enumerate}
    \item 对于段落中每一个单词与问题计算得到的注意力结果,将它与前一层的这个单词的上下文表示连接然后流向后面的建模层
    \item 利用一种无记忆的机制,虽然类似于Bahdanau注意力\upcite{neural machine translation by jointly learning to align and translate},
对于段落的每一个时间步都会计算注意力,但是每一步计算的注意力是仅仅是关于段落和问题当前时刻的单词,与前面的时刻或后面的时刻计算的注意力都无关
    \item 不像之前的论文\upcite{Machine comprehension using match-lstm and answer pointer}仅仅
计算段落对问题的注意力,而忽视了问题对段落的注意力。
\end{enumerate}

之前的论文注意力机制大多借鉴于Bahdanau注意力\upcite{neural machine translation by jointly learning to align and translate},这种注意力运算机制的特点
就是说将段落中每一个单词的上下文表示和问题的上下文表示
做注意力的运算,计算的结果作为下一个单词做注意力计算的输入。但是这种计算方式显然前一时间步计算得到的注意力的结果会影响
下一时间步注意力的计算,也就是说这种注意力的计算方式在时间上来讲是动态的。而BiDAF在计算注意力时对于段落中
当前时间步的单词和问题计算注意力的结果不会参与下一时间步的单词计算注意力。这种机制简化了注意力的计算流程,使得注意力流层和后面的
建模层分隔开来,注意力流层会更加的专注于计算注意力,而且计算先前时间步计算的注意力出现偏差,也不会影响后面计算注意力。

具体的,输入给注意力流层的是上下文嵌入表示层的输出$C$和$Q$,其中$C\in R^{2d\times T}$是整个段落的上下文表示,
$Q\in R^{2d\times J}$是问题的上下文表示,$T$代表段落的长度,$J$代表问题的长度。
首先计算相似性矩阵$S$:
\begin{gather}
    S_{tj}=\alpha(C_{:t},Q_{:j}) \in R \notag \\ 
    \alpha(c,q)=w_{(S)}^T[c;q;c\circ q] \notag
\end{gather}

$\alpha$是编码两个向量$C$和$Q$相关性的函数,$w_{(S)}^T\in R^{6d}$是可训练的权重参数,$\circ$表示点乘。
$S\in R^{T\times J},S_{t:}\in R^{J},S_{:j}\in R_{T}$,每一行表示的段落中的一个单词对问题的相关度,每一列
表示的是问题的一个单词对整个段落的相关度。
接下来利用$S$计算两个方向的注意力,对于段落对问题的注意力C2Q,计算方式如下:
\begin{gather}
    a_t=softmax(S_{t:}) \notag\\
    \widetilde{Q}_{:t}=\sum_{j=1}^{J}a_{tj}Q_{:j}\notag
\end{gather}

$\widetilde{C}\in R^{2d\times T}$,Q2C的注意力的计算方式如下:
\begin{gather}
    b=softmax(max_{col}(S))\in R^T \notag \\
    \widetilde{q}=\sum_{t=1}^{T}b_tC_{:t} \notag 
\end{gather}
$S$的第$i$行的最大值那一列$j$表示的是段落中第$i$个单词和问题中第$j
$个单词之间的相关度最高,所以$\widetilde{q}$表示的就是段落中所有关于问题相关度最高的
单词的加权求和,这种计算方式略去了那些段落中关于问题不重要的单词。然后将$\widetilde{q}$在列的维度上
重复$T$次得到$\widetilde{Q}\in R^{2d\times T}$。

最后将计算的注意力结果$\widetilde{Q},\widetilde{C}$以及段落的上下文表示
按照$[P;\widetilde{P};p\circ\widetilde{P};P\circ\widetilde{Q}]\in R^{8d\times T}$这种方式连接,作为下一层建模层的输入。
可以看出计算出来的注意力流向了下一层,避免由于过早的把段落和文本概括为一个固定长度的向量导致信息的损失。

\textbf{\songti \zihao{5} R-Net}
文献\upcite{RNet}提出一种门控的自匹配网络。模型由四部分构成。

\begin{figure}[htbp]
    \centering
    \includegraphics[width=1.0\linewidth]{rnet.png}
    \caption{Gated Self-Matching Networks structure overview.}
\end{figure}

第一部分利用双向GRU来编码文章和问题的词嵌入以及字符嵌入获得文章和问题中每一个单词的上下文表示,字符嵌入有助于解决那些未登录词典的词嵌入问题。
如图2所示,$u_1^Q,u_2^Q,\cdots,u_m^Q$表示经过BiGRU编码后的问题的上下文表示,$u_1^P,u_2^P,\cdots,u_n^P$表示文章的
上下文表示。

第二部分采用一种基于注意力的门控循环神经网络,这是基于注意力的循环神经网络的一种变体\upcite{neural machine translation by jointly learning to align and translate},
通过引入一个额外的门控单元放大那些在文章中与问题相关的部分的重要性,减小与问题无关的那部分的重要性。这种门控的思想
体现出一篇文章中只有其中一部分是与问题有关的。具体的计算方式如下:
\begin{gather}
    s_j^t=v^T\tanh(W_u^Qu_j^Q+W_u^Pu_t^P+W_v^Pv_{t-1}^P) \notag \\
    a_i^t=\exp(s_i^t)/\sum_{j=1}^{m}\exp(s_j^t)\notag \\
    c_t=\sum_{i=1}^{m}a_i^tu_i^Q \notag
\end{gather}
其中$u_j^Q$是问题的第$j$个单词的上下文表示,$u_t^P$是文章的第$t$个单词的上下文表示。$a_i^t$是
文章的第$t$个单词与问题的第$i$个单词之间的相关度。
$c_t$是将计算得到的注意力权重对整个问题加权求和得到的注意力结果。
$v_{t-1}^P$是当前层的循环神经网络上一时刻的隐藏状态。
\begin{gather}
    v_t^P=RNN(v_{t-1}^P,[u_t^P,c_t]^{*}) \notag \\
    [u_t^P,c_t]^{*}=g_t\circ [u_t^P,c_t] \notag \\
    g_t=\text{sigmoid}(W_g[u_t^P,c_t]) \notag 
\end{gather}
其中$g_t$为门控单元。

第三部分采用自匹配机制,虽然上一层的输出$v_1^P,v_2^P,\cdots,v_n^P$已经获得了关注问题的文章表示,但是一方面在SQuAD数据集中某些问题和文章
在语法和词汇上略有差异,另一方面循环神经网络实际上只能记忆有限的上下文信息,
从而导致答案中单词可能会忽视其周围单词的重要信息,
因此利用自匹配层对关注问题的文章表示做自注意力的计算。具体计算方式如下:
\begin{gather}
    s_j^t=v^T\tanh(W_v^Pv_j^P+W_v^{\widetilde{P}}v_t^P)\notag \\
    a_i^t=\exp(s_i^t)/\sum_{j=1}^{n}\exp(s_j^t)\notag \\
    c_t=\sum_{i=1}^{n}a_i^tv_i^P \notag \\    
    h_t^P=\text{BiRNN}(h_{t-1}^P,[v_t^P,c_t]^{*}) \notag \\
    [v_t^P,c_t]^{*}=g_t\circ [v_t^P,c_t] \notag \\
    g_t=\text{sigmoid}(W_g[v_t^P,c_t])\notag
\end{gather}\notag


第四部分是输出层,类似于Wang and Jiang\upcite{Machine comprehension using match-lstm and answer pointer}采用
Pointer Networks\upcite{Pointer Networks}来预测答案的起始位置和终止位置。在预测答案的起始位置时,输出层的初始状态
采用的是对问题中单词的上下文表示向量的加权求和。

\textbf{\zihao{5} QANet}
先前的模型大多使用循环神经网络来编码句子语义以及基于循环神经网络的注意力机制对文章和问题做交互。但是众所周知循环神经网络由于
序列式的特性使得训练是非常耗时的。另一方面循环神经网络由于梯度消失问题难以解决长距离依赖的情况。因此QANet\upcite{QANet}中完全
舍弃用循环神经网络作为编码器,而采用卷积神经网络与自注意力机制结合的方式再加入到transformers\upcite{Transformer}结
构中作为模型的编码器。值得一提的是虽然transformers结构已经被广泛的应用在各种NLP任务中,但是利用卷积与自注意机制的结合
是一种新的结构而且实验表明比单独的利用自注意力机制要高出2.7F1值。QANet的解码器结构与transformer的结构如图。
可以看出区别在于QANet中的编码器在输入自注意力层之前要经过几层卷积层。这里的卷积方式与传统的卷积方式不同,
采用的是深度可分离卷积\upcite{Deepwise Separable convolution}。这种卷积操作可以减少卷积层
中参数的数量。传统的卷积操作是在
每一个通道上利用卷积核同步卷积所有通道,而深度可分离卷积将传统的卷积步骤分为两个步骤。第一个步骤是
Depthwise卷积,也就是在这一步骤中每一个卷积核只负责一个通道,一个通道只由一个卷积核卷积,这样使得输入数据的特征在深度上是分离的。
第二个步骤是Pointwise卷积,将第一步输出数据在通道这一维度连接起来卷积。
完整的结构如图所示。

在输入嵌入层中一个单词的嵌入表示是词嵌入和字符嵌入的连接,其中词嵌入采用GloVe预训练的词向量\upcite{GloVe},字符嵌入采用\upcite{charCNN}
中的方式,利用卷积操作提取每一个单词的字符特征。然后将嵌入后的特征通过两层高速网络\upcite{Highway Network},输出的结果作为
嵌入编码层的输入。两个嵌入编码层的输出送入到注意力层做交互,这一层相似度矩阵的计算方式类似于BiDAF\upcite{Bidirectional attention flow for machine comprehension}
,在计算问题对文章的注意力时采用和DCN\upcite{Dynamic coattention networks for question answering}相似的计算方式,
即将文章对问题的注意力也同时考虑进去。注意力层的输出采用\upcite{Bidirectional attention flow for machine comprehension}
的思想,将文章对问题的注意力流向后面的层以减少过早加权和后的损失。然后将输出通过几层模型编码层,模型编码层后三层的输出分别作为
输入层的输入来分别的预测答案的起始位置的概率和终止位置的概率。输出层的设计是根据具体任务的,通过修改输出层,整个QANet也可以
用到其它类型的问答任务中。

数据增强

%
%
%\subsection{答案预测层的设计}\label{output}
如前所述,答案预测层的设计要依据答案的形式而设计,下面介绍
各个模型在四类不同的MRC任务上的输出层设计。

1)填空型:这类任务答案的形式是预测问题中缺失的单词,而且缺失的答案来源于文章中。文献\cite{Hermann}
最早提出将问题的语义向量与关注了问题的文章的语义向量拼接成一个向量然后映射到整个词典中预测那个缺失的单词。
这种方法存在的一个问题就是不能够确保预测的单词一定是文章中的词汇,因为它是将最后的输出向量映射到整个词典中,
这就使得模型的预测准确率受到影响。指针网络(Pointer networks\upcite{Ptr})模型由seq2seq模型演变而来,主要就是
为了解决输出源自于输入的问题,实现方式是利用计算的注意力的权重分布直接输出预测结果,而这种机制正适合
填空型任务以及片段选择型任务。
文献\cite{ASR}提出AS Reader模型正是受到指针网络的启发,对于计算得到的注意力权重分布,将其中相同单词的注意力
权值相加,最后输出具有最大权值的单词最为答案。
填空型任务模型的损失函数可以写
为$L(\theta)=-\displaystyle\frac{1}{N}\sum_{i=1}^{N}\log P_{y_i}$。
其中$\theta$为模型参数,$N$代表样本数目,$y_i$表示文章中第$i$个样本的文章中标准答案的位置。

2)多项选择型:这类任务是从多个候选答案选项中选择正确的选项。处理这种任务最简单的一种方式就是计算模型输出后的
文章的语义信息和选项之间的相似程度,相似程度最高的作为预测的选项,从而将问题变化为句子之间的语义匹配问题。
文献\cite{Co-matching}提出将问题、文章、选项一起方法模型中做交互计算输出一个向量作为输出层的输入,
输出层采用简单的输出维度是1的全连接层,输出的值代表模型对这个选型的打分值,其它的选项类似的处理,值最高的选项作为预测的答案。
最后对所有选项的打分值
做归一化作为模型的损失函数。
多项选择型任务模型的损失函数可以写为
$L(\theta)=-\displaystyle\frac{1}{N}\sum_{i=1}^{N}\sum_{j=1}^{m}\log P_{y_{j}^i}$。
其中$\theta$为模型参数,$N$代表样本数目,$m$代表选项个数,$y_{j}^i$表示第$i$个样本中第$j$个选项是正确答案。

3)片段选择型:这类任务是从文章中提取出来一段连续的单词作为答案,虽然类似于填空型任务输出来源自输入的性质,但是
不像填空型任务仅仅只是预测一个单词。因此填空型
任务答案输出层的设计不能直接用来作为片段选择型任务答案预测层。
由于提取文本的长度不固定,使得这一任务更具有挑战性。
文献\cite{MatchLSTM}受到指针网络的启发提出了两种基于指针网络的输出模型,第一种是序列式模型,利用指针网络以一种序列式的形式生成答案的每一个位置,处理过程类似于seq2seq模型的解码过程,
这种模型下答案的每一个单词可能出现在文本段落的任何一个位置,
这是因为指针网络并没有要求从输入中选择的输出具有连续性。由于答案的长度不固定,因此在段落中设置一个特殊的
位置表示答案的终止点,当预测到这个位置时终止答案的生成。
第二种是边界式模型,不同于序列式模型那样序列的生成答案的每一个位置,由于要预测的答案是一段连续的单词的组合,
因此可以利用指针网络仅仅预测答案的起始位置和终止位置。所预测答案的概率
是预测这两个位置概率的乘积,这种方式相比于
第一种更加的简单而且测试结果表明更加高效。
%\begin{table*}[ht]
%    \centering
%    \caption{模型对比(acc代表准确率)}
%    \vspace{10pt}
%    \begin{tabular}{l c c c}
%        \toprule
%        \multirow{2}{*}{模型}&SQuAD 1.1\upcite{SQuAD1}& SQuAD 2.0\upcite{SQuAD2} & RACE\upcite{RACE}\\
%        \cmidrule(lr){2-2} \cmidrule(lr){3-3} \cmidrule(lr){4-4}
%        &EM/F1&EM/F1& acc \\
%        \midrule
%        Match-LSTM\upcite{MatchLSTM}& 64.7/73.7 & -&-\\
%        \midrule
%        DCN\upcite{Dynamic coattention networks for question answering}& 66.2/75.9 &-&-\\
%        \midrule
%        ReasoNet\upcite{Reasonet}&70.6/79.4 &-&-\\
%        \midrule
%        BiDAF\upcite{Bidirectional attention flow for machine comprehension}&68.0/77.3 &-&-\\
%        \midrule
%        R-Net\upcite{RNet}&72.3/80.7 &-&-\\
%        \midrule
%        QANet(data augmentation x3)\upcite{QANet}& 76.2/84.6&-&-\\
%        \midrule
%        ELMo+BiDAF\upcite{ELMo}&78.6/85.8&-&-\\
%        \midrule
%        GPT-v1\upcite{GPT}&-&-&59.0\\
%        \midrule
%        $\text{BERT}_{large}$(+TriviaQA)\upcite{BERT}& 85.1/91.8 &80.0/83.1&72.0\\
%        \midrule
%        XLNet\upcite{XLNet}&-&86.4/89.1 &81.8\\
%        \midrule
%        DCMN+\upcite{DCMN+}&-&-&82.8\\
%        \midrule
%        RoBERTa\upcite{RoBERTa}&-&86.8/89.8 &83.2\\
%        \midrule
%        ALBERT\upcite{ALBERT}&-&88.1/90.9 &86.5\\
%        \midrule
%        Retro-Reader over ALBERT\upcite{Retrospective}&-&88.1/91.4&-\\
%        \bottomrule
%    \end{tabular}
%\end{table*}

边界式模型的这种设计思想也被后来很多MRC模型采纳。如R-Net\upcite{RNet}采用几乎一样的边界式
模型,只不过解码端的初始状态
采用的是对问题的语义向量加权求和。尽管边界式模型简单有效,但是在文章中可能有些文本片段与标准答案相似,比如初始位置一样,
那么边界式模型有可能陷入局部极值的情况从而提取错误的文本片段。为了处理这个问题,DCN\upcite{DCN}
模型提出一种动态迭代的指针网络作为解码端,利用上一次预测的答案的起始位置和终止位置以及解码端当前的状态来重新评估
下一次预测答案的起始位置和终止位置。多次迭代后选取所有迭代次数中概率最大的情形作为预测答案。
片段选择任务模型的损失函数可以写为
$L(\theta)=-\displaystyle\frac{1}{N}\sum_{i=1}^{N}\log P_{y_i^s}^S+\log P_{y_i^e}^E$。
其中$\theta$为模型参数,$N$代表样本数目,$y_i^s$表示第$i$个样本中标准答案的起始位置在文章中的位置,
$y_i^e$表示第$i$个样本中标准答案的终止位置在文章中的位置。
如果考虑到不可回答的问题,最简单的方式是额外在输出层加上一个输出维度是1的全连接层。
此时的损失函数可以写为
$L(\theta)=-\displaystyle\frac{1}{N}\sum_{i=1}^{N}(\log P_{y_i^s}^S+\log P_{y_i^e}^E)+\log P_{y_i^u}^U$。
其中$y_i^u$表示第$i$个样本中的问题是不可回答的问题。关于带有不可回答问题的阅读理解任务细节见\ref{unknown}节。

4)自由答案型:这类任务的答案形式已经不再是原文中某段文本,而是需要根据文章和问题生成符合语法规范的文本。
这类任务对答案生成模块的能力要求较高。处理生成任务典型的架构是seq2seq模型,文献\cite{PGNet}提出
一种指针生成网络模型(Pointer-Generator Network,PGNet)。这个模型最早提出来用在文本摘要领域,模型结合了seq2seq的
生成机制以及指针网络的拷贝机制,使得模型既能从词典中生成单词又能在原文中拷贝单词,实验结果表明该模型的效果优于传统的seq2seq模型。
% 文献\cite{CoQA}采用三种模型在在CoQA数据集上进行实验。
% 第一种是传统的seq2seq模型,第二种是Pointer-Generator Network\cite{PGNet},简称PGNet。
% 这个模型最早提出来用在文本摘要领域,模型结合了seq2seq的生成机制以及
% 指针网络的拷贝机制,使得模型既能生成单词又能在原文中拷贝单词,实验结果表明该模型的效果优于传统的seq2seq模型。
% 第三种是DrQA+PGNet模型,其中DrQA\cite{DrQA}是类似于上述
% BiDAF\upcite{Bidirectional attention flow for machine comprehension},
% RNet\upcite{RNet}等模型的片段抽取模块。整个模型的思想是
% 先利用片段提取模块从文章中提取中与问题最相关的一段文本,然后利用答案生成模块在这个被抽取出来的文本上
% 生成答案。这种(提取模块+生成模块)模型在其它的生成模型中如SNet\upcite{SNet}等广泛使用。
% 实验结果也表明这种模型的效果是最好的。








\begin{table}[ht]
	\centering
	\caption{模型对比(acc代表准确率)\\ Table 8 Model comparison(acc represents accuracy)}
	%\vspace{10pt}
	\begin{tabular}{l c c c}
		\toprule
		\multirow{2}{*}{模型}&SQuAD 1.1\upcite{SQuAD1}& SQuAD 2.0\upcite{SQuAD2} & RACE\upcite{RACE}\\
		\cmidrule(lr){2-2} \cmidrule(lr){3-3} \cmidrule(lr){4-4}
		&EM/F1&EM/F1& acc \\
		\midrule
		Match-LSTM\upcite{MatchLSTM}& 64.7/73.7 & -&-\\
		\midrule
		DCN\upcite{DCN}& 66.2/75.9 &-&-\\
		\midrule
		ReasoNet\upcite{Reasonet}&70.6/79.4 &-&-\\
		\midrule
		BiDAF\upcite{BiDAF}&68.0/77.3 &-&-\\
		\midrule
		R-Net\upcite{RNet}&72.3/80.7 &-&-\\
		\midrule
		QANet\upcite{QANet}& 76.2/84.6&-&-\\
		\midrule
		ELMo+BiDAF\upcite{ELMo}&78.6/85.8&-&-\\
		\midrule
		GPT-v1\upcite{GPT}&-&-&59.0\\
		\midrule
		$\text{BERT}_{large}$\upcite{BERT}& 85.1/91.8 &80.0/83.1&72.0\\
		\midrule
		XLNet\upcite{XLNet}&89.9/95.1&86.4/89.1 &81.8\\
		\midrule
		RoBERTa\upcite{RoBERTa}&-&86.8/89.8 &83.2\\
		\midrule
		ALBERT\upcite{ALBERT}&-&88.1/90.9 &86.5\\
		\bottomrule
	\end{tabular}
\end{table}

%\subsection{小结}
%本章首先介绍了神经机器阅读理解模型的结构,
%涉及的技术如注意力机制以及相关的模型。
%然后介绍了目前流行的几种预训练模型,最后对于不同的MRC任务分别介绍了各自输出层的
%设计。
%
%在预训练模型的基础上利用具体任务的数据微调模型,
%即只需要稍微添加简单的输出层即可达到很好的效果。但是预训练模型只是给了更好的初始化参数,
%如果想要进一步提升模型的性能还需要在其基础上根据具体的任务设计一个更好的模型。
%%如Zhang等人\upcite{DCMN}
%%利用BERT\upcite{BERT}和XLNet\upcite{XLNet}作为编码器同时采用文献\cite{Co-matching}提出的co-matching方法提出了DCMN模型,在RACE\upcite{RACE}
%%数据集上达到了很高的准确率。而后在DCMN的基础上引入选项交互和段落选择任务使得
%如何利用预训练模型结合具体任务改进模型的结构是至关重要的。表8对比了本章所介绍的模型在三个常用数据集上的表现,可以看到基于预训练
%模型的效果要显著的优于其它模型,此外在其它数据集上排名靠前的模型几乎都是基于预训练模型的。
%\end{multicols}
%\begin{table}[ht]
%	\centering
%	\caption{模型对比(acc代表准确率)\\ Table 8 Model comparison(acc represents accuracy)}
%	%\vspace{10pt}
%	\begin{tabular}{l c c c}
%		\toprule
%		\multirow{2}{*}{模型}&SQuAD 1.1\upcite{SQuAD1}& SQuAD 2.0\upcite{SQuAD2} & RACE\upcite{RACE}\\
%		\cmidrule(lr){2-2} \cmidrule(lr){3-3} \cmidrule(lr){4-4}
%		&EM/F1&EM/F1& acc \\
%		\midrule
%		Match-LSTM\upcite{MatchLSTM}& 64.7/73.7 & -&-\\
%		\midrule
%		DCN\upcite{DCN}& 66.2/75.9 &-&-\\
%		\midrule
%		ReasoNet\upcite{Reasonet}&70.6/79.4 &-&-\\
%		\midrule
%		BiDAF\upcite{BiDAF}&68.0/77.3 &-&-\\
%		\midrule
%		R-Net\upcite{RNet}&72.3/80.7 &-&-\\
%		\midrule
%		QANet\upcite{QANet}& 76.2/84.6&-&-\\
%		\midrule
%		ELMo+BiDAF\upcite{ELMo}&78.6/85.8&-&-\\
%		\midrule
%		GPT-v1\upcite{GPT}&-&-&59.0\\
%		\midrule
%		$\text{BERT}_{large}$\upcite{BERT}& 85.1/91.8 &80.0/83.1&72.0\\
%		\midrule
%		XLNet\upcite{XLNet}&89.9/95.1&86.4/89.1 &81.8\\
%		\midrule
%		RoBERTa\upcite{RoBERTa}&-&86.8/89.8 &83.2\\
%		\midrule
%		ALBERT\upcite{ALBERT}&-&88.1/90.9 &86.5\\
%		\bottomrule
%	\end{tabular}
%\end{table}

%\input{new_trend.tex}
%
\section{讨论}
%本章主要讨论MRC领域的发展历史和目前MRC领域存在的主要问题
%本章介绍了MRC任务的定义以及根据答案类型的不同划分四种任务并且介绍了每一个任务的形式以及相关的数据集
%,同时简述了每一个任务下衡量模型性能所用的评估指标。
%下表5从数据集来源及规模,文章、问题以及答案的类型角度上对比了本章所介绍的所有数据集
\subsection{回顾MRC的发展历史}
从MRC数据集的发展角度看,从最简单的填空型数据集,到抽取型数据集再到复杂的需要从多段落中归纳答案的数据集等
每一个新的数据集都会在原有数据集的基础上增加各种各样的难度,从而不得不设计更加优秀的模型处理这些任务
。
从模型内部各个层次的发展角度上看,
%词嵌入层从早期最常用的预训练静态词向量Word2Vec\upcite{word2vec}及GloVe\upcite{GloVe}发展到基于上下文的动态词嵌入技术ELMo\upcite{ELMo},编码层从最常用的循环神经网络及其变体(如LSTM\upcite{LSTM}等)发展到更强大的特征提取器transformer\upcite{Transformer}。
嵌入层和编码层在预训练模型出现之前并没有较大的变动。各个模型主要集中在交互层的注意力机制上,为了获得更全面的交互信息从单向发展到双向,为了可以达到多步推理的目的从单跳结构发展到多跳结构。
而对于更加复杂的阅读理解任务时,模型整体发生了较大的改动。面对无答案的阅读理解任务设计答案验证模块来验证问题是否可以回答,面对多段落型阅读理解任务设计段落选择模块减小答案搜索范围,面对对话形式的阅读理解任务通过将之前轮的对话信息以文本拼接或者信息流动的方式使得模型在回答当前轮问题的时候可以关注到之前的对话信息。
%Match-LSTM\upcite{MatchLSTM}等改进到双向(如BiDAF\upcite{BiDAF}、DCN\upcite{DCN}等),单层(如)

%在预训练模型出现后不仅仅在MRC任务上在其它的NLP任务上模型的结构都发生了较大的变化,
%%在模型的基础上利用具体任务的数据微调模型即可达到很好的效果,
%如何训练出一个性能更加强大的预训练模型成为近年来NLP领域的研究热点。
%但是预训练模型只是给了更好的初始化参数,
%如果想要进一步提升模型的性能还需要在其基础上根据具体的任务设计一个更好的模型。
%如Zhang等人\upcite{DCMN}
%利用BERT\upcite{BERT}和XLNet\upcite{XLNet}作为编码器同时采用文献\cite{Co-matching}提出的co-matching方法提出了DCMN模型,在RACE\upcite{RACE}
%数据集上达到了很高的准确率。
%而后在DCMN的基础上引入选项交互和段落选择任务使得
%如何利用预训练模型结合具体任务改进模型的结构是至关重要的。
%表8对比了本章所介绍的模型在三个常用数据集上的表现,可以看到基于预训练
%模型的效果要显著的优于其它模型,此外在其它数据集上排名靠前的模型几乎都是基于预训练模型的。

%带有不可回答问题的任务,多段落型任务以及对话型问答任务,这三个任务相比于之前阅读理解任务都有各自的特点和难点所在。实验表明之前的经典模型在比较简单的数据集上已经达到了很好的效果,然而迁移到这几个复杂的阅读理解任务上模型效果显著下降。如何设计性能更加强大的模型来处理更加复杂的任务还需要大量的研究。



%正是这些越来越具有挑战性的数据集的发布,极大的推动了MRC领域的发展。


\subsection{分析MRC面临的主要问题}
预训练模型出现后基于神经网络的MRC模型的性能再一次都取得了显著的进步,但是目前MRC领域还有一些问题需要解决。

\textbf{模型缺乏推理能力} \quad
如前面所述,基于注意力机制的匹配模型大多是浅层的语义匹配模型,基于多跳结构的推理模式还过于单一,这些均没有形成深层次的阅读理解模型。Jia等人\upcite{Jia}在SQuAD数据集的基础上设计对抗样本来测试MRC模型,在文章中加入人工设计的句子,这些句子与问题相似但是与答案无关以此用来误导模型。这些误导的句子对人来说没有什么影响然而实验证明几乎所有模型的准确率都显著下降,这表明模型靠的是关键词匹配方式寻找答案而真正的阅读理解能力和推理能力很弱,让模型具有较强的推理能力至关重要。

\textbf{答案生成技术不足} \quad
生成答案的技术还需要进一步提升,回顾目前机器阅读理解领域的数据集以及相应的模型,
大多集中于片段抽取式问答且模型准确度很高,
而对于自由答案型这种需要生成答案的模型效果很差,有些模型直接将生成式问题转为抽取式问题效果反而优于生成式的做法,主要原因
在于生成答案模块的效果不够好。

\textbf{模型缺乏外部知识} \quad
目前很少有机器阅读理解模型融合外部知识,都是直接根据给定的文档回答相关的问题,而人在阅读一篇文章的时候对这篇文章的理解程度和他已经掌握的知识水平有很大关系。因此如果将外部知识源融入模型中,模型的性能大概率会显著提高。

\subsection{探索MRC未来的研究方向}

\textbf{如何设计高质量的数据集} \quad
前面介绍了很多MRC领域常用的数据集,也同时分析了各个数据集的难度、特点、规模等。虽然目前的数据集已经不像之前的数据集模型仅仅通过浅层的匹配机制就能够达到很高的效果,但是一个高质量的机器阅读理解数据集不应该是普通的大规模、高难度的特点,它要能够准确的考察模型是否具有人类阅读能力,包括推理能力,常识能力等。

机器阅读理解赋予了计算机阅读理解文本的能力,在搜索、对话、医疗以及教育领域都有着广阔的应用空间,未来的研究方向有以下几点值得关注:

1)目前机器阅读理解领域的数据集大多是通用领域方向,而设计专业领域数据集也尤为重要。比如通过利用机器阅读理解技术分析产品说明文档和用户的问题语义从而解决用户对产品的问题达到智能客服服务,或者在医疗诊断中分析大量病例和知识库提供智能医疗服务。


2)目前机器阅读理解主要集中于非结构化的文本领域,而
还有许多其它结构、不同模态的数据如表格、视频、音频、图片等,相关的研究方向如数据库问答,视觉问答等。
多模态阅读理解模型是未来的机器阅读理解发展方向之一。

%3)目前很少有机器阅读理解模型融合外部知识,都是直接根据给定的文档回答相关的问题,而人在阅读一篇文章的时候对这篇文章的理解程度和他已经掌握的知识水平有很大关系。因此如果将外部知识源融入模型中那么模型的性能大概率会显著提高。





















%
%\begin{thebibliography}{99}
%    \bibitem{lstm}Sepp Hochreiter and Jürgen Schmidhuber. Long short-term memory. Neural computation, 9(8):
%    1735–1780, 1997.
%
%    % \bibitem{BLEU}Kishore Papineni, Salim Roukos, Todd Ward, and Wei-Jing Zhu. Bleu: a method for au-
%    % tomatic evaluation of machine translation. In Proceedings of the 40th Annual Meeting on
%    % Association for Computational Linguistics, pages 311–318. Association for Computational
%    % Linguistics, 2002.
%    \bibitem{ROUGE}Chin-Yew Lin. Rouge: A package for automatic evaluation of summaries. Text Summariza-
%    tion Branches Out, 2004.
%
%    \bibitem{Maxout Network}Goodfellow, I.J., Warde-Farley, D., Mirza, M., Courville, A., Bengio, Y.: Maxout networks.
%    arXiv preprint arXiv:1302.4389 (2013)
%
%    \bibitem{word2vec}Mikolov T,Sutskever I,Chen K,et al.Distributed represen-
%    tations of words and phrases and their compositionality[C]//
%    Advances in Neural Information Processing Systems,
%    2013:3111-3119.
%
%    \bibitem{GRU}Junyoung Chung, Caglar Gulcehre,
%    Kyunghyun Cho, and Yoshua Bengio. 2014. Empir-
%    ical Evaluation of Gated Recurrent Neural Networks
%    on Sequence Modeling. arXiv, pages 1–9.
%
%    \bibitem{GloVe}Jeffrey Pennington, Richard Socher, and Christopher D. Manning. Glove: Global vectors for
%    word//w representation. In Empirical Methods in Natural Language Processing (EMNLP), pp.
%    1532–1543, 2014.
%
%    \bibitem{neural machine translation by jointly learning to align and translate}Dzmitry Bahdanau, Kyunghyun Cho, and Yoshua Bengio. Neural machine translation by jointly
%    learning to align and translate. ICLR, 2015.
%
%    \bibitem{memory network}J. Weston, S. Chopra, and A. Bordes. Memory networks. In International Conference on
%    Learning Representations (ICLR), 2015.
%
%    \bibitem{Highway Network}Srivastava, R.K., Greff, K., Schmidhuber, J.: Highway networks.
%    arXiv preprint
%    arXiv:1505.00387 (2015)
%
%    \bibitem{MemN2N}Sainbayar Sukhbaatar, Jason Weston, Rob Fergus, et al. End-to-end memory networks. In
%    Advances in neural information processing systems, pages 2440–2448, 2015.
%
%    \bibitem{Teaching Machines to Read and Comprehend}Hermann, K.M., Kocisky, T., Grefenstette, E., Espeholt, L., Kay, W., Suleyman, M., Blun-
%    som, P.: Teaching machines to read and comprehend. In: Advances in Neural Information
%    Processing Systems, pp. 1693–1701 (2015)
%
%    \bibitem{AR}Danqi Chen, Jason Bolton, and Christopher D Manning. A thorough examination of the
%    cnn/daily mail reading comprehension task. In Proceedings of the 54th Annual Meeting of
%    the Association for Computational Linguistics (Volume 1: Long Papers), volume 1, pages
%    2358–2367, 2016.
%
%    \bibitem{ASR}Rudolf Kadlec, Martin Schmid, Ondřej Bajgar, and Jan Kleindienst. Text understanding
%    with the attention sum reader network. In Proceedings of the 54th Annual Meeting of the
%    Association for Computational Linguistics (Volume 1: Long Papers), volume 1, pages 908–
%    918, 2016.
%
%    \bibitem{IAReader}Alessandro Sordoni, Philip Bachman, Adam Trischler, and Yoshua Bengio. Iterative alternat-
%    ing neural attention for machine reading. arXiv preprint arXiv:1606.02245, 2016.
%
%    \bibitem{SQuAD1}Pranav Rajpurkar, Jian Zhang, Konstantin Lopyrev, and Percy Liang. SQuAD: 100,000+ questions
%    for machine comprehension of text. In Proceedings of the Conference on Empirical Methods in
%    Natural Language Processing, 2016.
%
%    \bibitem{sentinel vector}Merity, S., Xiong, C., Bradbury, J., Socher, R.: Pointer sentinel mixture models. arXiv
%    preprint arXiv:1609.07843 (2016)
%
%    \bibitem{Deepwise Separable convolution}François Chollet. Xception: Deep learning with depthwise separable convolutions.
%    abs/1610.02357, 2016.
%
%    \bibitem{Machine comprehension using match-lstm and answer pointer}Wang, S., Jiang, J.: Machine comprehension using match-lstm and answer pointer. arXiv
%    preprint arXiv:1608.07905 (2016)
%
%    \bibitem{Dynamic coattention networks for question answering}Xiong, C., Zhong, V., Socher, R.: Dynamic coattention networks for question answering.
%    arXiv preprint arXiv:1611.01604 (2016)
%
%    \bibitem{Bidirectional attention flow for machine comprehension}Seo, M., Kembhavi, A., Farhadi, A., Hajishirzi, H.: Bidirectional attention flow for machine
%    comprehension. arXiv preprint arXiv:1611.01603 (2016)
%    
%    \bibitem{RNet}Wang, W., Yang, N., Wei, F., Chang, B., Zhou, M.: Gated self-matching networks for
%    reading comprehension and question answering. In: Proceedings of the 55th Annual Meet-
%    ing of the Association for Computational Linguistics (Volume 1: Long Papers), vol. 1, pp.
%    189–198 (2017)
%
%    \bibitem{Reasonet}Shen, Y., Huang, P.S., Gao, J., Chen, W.: Reasonet: Learning to stop reading in machine
%    comprehension. In: Proceedings of the 23rd ACM SIGKDD International Conference on
%    Knowledge Discovery and Data Mining, pp. 1047–1055. ACM (2017)
%
%    \bibitem{GAReader}Bhuwan Dhingra, Hanxiao Liu, Zhilin Yang, William Cohen, and Ruslan Salakhutdinov.
%    Gated-attention readers for text comprehension. In Proceedings of the 55th Annual Meeting
%    of the Association for Computational Linguistics (Volume 1: Long Papers), pages 1832–1846,
%    2017.
%
%    \bibitem{PGNet}Abigail See, Peter J. Liu, and Christopher D.
%    Manning. 2017. Get to the point: Summa-
%    rization with pointer-generator networks. In
%    Annual Meeting of the Association for Compu-
%    tational Linguistics (ACL), pages 1073–1083.
%    Vancouver, Canada.
%
%    \bibitem{TriviaQA}Mandar Joshi, Eunsol Choi, Daniel Weld, and Luke Zettlemoyer. Triviaqa: A large scale
%    distantly supervised challenge dataset for reading comprehension. In Proceedings of the 55th
%    Annual Meeting of the Association for Computational Linguistics (Volume 1: Long Papers),
%    volume 1, pages 1601–1611, 2017.
%
%    \bibitem{DrQA}Danqi Chen, Adam Fisch, Jason Weston, and
%    Antoine Bordes. 2017. Reading Wikipedia
%    to answer open-domain questions. In Asso-
%    ciation for Computational Linguistics (ACL),
%    pages 1870–1879. Vancouver, Canada.
%
%    \bibitem{FusionNet}Hsin-Yuan Huang, Chenguang Zhu, Yelong Shen, Weizhu Chen.
%    FusionNet: Fusing via Fully-Aware Attention with Application to Machine Comprehension. In Proceedings of the 
%    Sixth International Conference on Learning Representations (ICLR), 2018.
%
%    \bibitem{MatchLSTM}Shuohang Wang and Jing Jiang. Learning natural language inference with LSTM. In Proceedings of
%    the Conference on the North American Chapter of the Association for Computational Linguistics,
%    2016.
%    \bibitem{Pointer Networks}Oriol Vinyals, Meire Fortunato, and Navdeep Jaitly. Pointer networks. In Proceedings of the Con-
%    ference on Advances in Neural Information Processing Systems, 2015.
%    \bibitem{RACE}Guokun Lai, Qizhe Xie, Hanxiao Liu, Yiming Yang, and Eduard Hovy. Race: Large-scale
%    reading comprehension dataset from examinations. In Proceedings of the 2017 Conference
%    on Empirical Methods in Natural Language Processing, pages 785–794, 2017.
%    \bibitem{SQuAD2}Rajpurkar, P., Jia, R., Liang, P.: Know what you don’t know: Unanswerable questions for
%    squad. arXiv preprint arXiv:1806.03822 (2018)
%    \bibitem{CBT}Hill, F., Bordes, A., Chopra, S., Weston, J.: The goldilocks principle: Reading children’s
%    books with explicit memory representations. arXiv preprint arXiv:1511.02301 (2015)
%    \bibitem{QANet}Yu, A.W., Dohan, D., Luong, M.T., Zhao, R., Chen, K., Norouzi, M., Le, Q.V.: Qanet:
%    Combining local convolution with global self-attention for reading comprehension. arXiv
%    preprint arXiv:1804.09541 (2018)
%
%    \bibitem{QuAC}Eunsol Choi, He He, Mohit Iyyer, Mark Yatskar, Wen-tau Yih, Yejin Choi, Percy Liang,
%    and Luke Zettlemoyer. Quac: Question answering in context. In Proceedings of the 2018
%    Conference on Empirical Methods in Natural Language Processing, pages 2174–2184, 2018.
%
%    \bibitem{DuReader}Wei He, Kai Liu, Jing Liu, Yajuan Lyu, Shiqi Zhao, Xinyan Xiao, Yuan Liu, Yizhong Wang,
%    Hua Wu, Qiaoqiao She, et al. Dureader: A chinese machine reading comprehension dataset
%    from real-world applications. In Proceedings of the Workshop on Machine Reading for Ques-
%    tion Answering, pages 37–46, 2018.
%
%    \bibitem{CoQA}Siva Reddy, Danqi Chen, and Christopher D Manning. Coqa: A conversational question answering
%    challenge. arXiv preprint arXiv:1808.07042, 2018.
%    \bibitem{MS marco}Nguyen, T., Rosenberg, M., Song, X., Gao, J., Tiwary, S., Majumder, R., Deng, L.:
%    Ms marco: A human generated machine reading comprehension dataset. arXiv preprint
%    arXiv:1611.09268 (2016)
%    \bibitem{NarrativeQA}Kočiskỳ, T., Schwarz, J., Blunsom, P., Dyer, C., Hermann, K.M., Melis, G., Grefenstette,
%    E.: The narrativeqa reading comprehension challenge. Transactions of the Association of
%    Computational Linguistics 6, 317–328 (2018)
%    \bibitem{Transformer}Ashish Vaswani, Noam Shazeer, Niki Parmar, Jakob Uszkoreit, Llion Jones, Aidan N Gomez,
%    Lukasz Kaiser, and Illia Polosukhin. Attention is all you need. In Neural Information Processing
%    Systems, 2017b.
%
%    \bibitem{ELMo}Peters, M.E., Neumann, M., Iyyer, M., Gardner, M., Clark, C., Lee, K., Zettlemoyer, L.:
%    Deep contextualized word representations. arXiv preprint arXiv:1802.05365 (2018)
%    \bibitem{GPT}Radford, A., Narasimhan, K., Salimans, T., Sutskever, I.: Improving language understand-
%    ing with unsupervised learning. Tech. rep., Technical report, OpenAI (2018)
%    \bibitem{BERT}Devlin, J., Chang, M.W., Lee, K., Toutanova, K.: Bert: Pre-training of deep bidirectional
%    transformers for language understanding. arXiv preprint arXiv:1810.04805 (2018)
%
%    \bibitem{XLNet}Zhilin Yang, Zihang Dai, Yiming Yang, Jaime Carbonell, Ruslan Salakhutdinov, and Quoc V
%    Le. XLNet: Generalized autoregressive pretraining for language understanding. arXiv preprint
%    arXiv:1906.08237, 2019.
%
%    \bibitem{Transformer-XL}Zihang Dai, Zhilin Yang, Yiming Yang, William W Cohen, Jaime Carbonell, Quoc V Le,
%    and Ruslan Salakhutdinov. Transformer-xl: Attentive language models beyond a fixed-length
%    context. arXiv preprint arXiv:1901.02860, 2019.
%
%    \bibitem{RoBERTa}Yinhan Liu, Myle Ott, Naman Goyal, Jingfei Du, Mandar Joshi, Danqi Chen, Omer Levy, Mike
%    Lewis, Luke Zettlemoyer, and Veselin Stoyanov. RoBERTa: A robustly optimized BERT pre-
%    training approach. arXiv preprint arXiv:1907.11692, 2019.
%
%    \bibitem{ALBERT}Zhenzhong Lan, Mingda Chen, Sebastian Goodman,
%    Kevin Gimpel, Piyush Sharma, and Radu Soricut.
%    2020. ALBERT: A lite BERT for self-supervised
%    learning of language representations. In ICLR.
%    \bibitem{VQACo}Jiasen Lu, Jianwei Yang, Dhruv Batra, and Devi Parikh. Hierarchical question-image co-attention
%    for visual question answering. arXiv preprint arXiv:1606.00061, 2016.
%    \bibitem{UNILM}Li Dong, Nan Yang, Wenhui Wang, Furu Wei, Xiaodong Liu,
%    Yu Wang, Jianfeng Gao, Ming Zhou, and Hsiao-Wuen Hon.
%    Unified language model pre-training for natural language un-
%    derstanding and generation. In NeurIPS, pages 13042–13054,
%    2019.
%    \bibitem{Co-matching}Shuohang Wang, Mo Yu, Jing Jiang, and Shiyu Chang.
%    2018. A Co-Matching Model for Multi-choice
%    Reading Comprehension. In Proceedings of the 56th
%    Annual Meeting of the Association for Computa-
%    tional Linguistics (Volume 2: Short Papers), pages
%    746–751. Association for Computational Linguis-
%    tics.
%
%    \bibitem{SNet}Chuanqi Tan, Furu Wei, Nan Yang, Bowen Du, Weifeng Lv, and Ming Zhou. S-net: From
%    answer extraction to answer synthesis for machine reading comprehension. In Thirty-Second
%    AAAI Conference on Artificial Intelligence, 2018.
%    \bibitem{DCMN}Shuailiang Zhang, Hai Zhao, Yuwei Wu, Zhuosheng Zhang, Xi Zhou, Xiang Zhou.
%    Dual Co-Matching Network for Multi-choice Reading Comprehension.
%    CoRR, abs/1901.09381
%
%    \bibitem{DCMN+}Shuailiang Zhang, Hai Zhao, Yuwei Wu, Zhuosheng Zhang, Xi Zhou, Xiang Zhou.
%    DCMN+: Dual Co-Matching Network for Multi-choice Reading Comprehension.
%
%
%    \bibitem{SDNet}Chenguang Zhu, Michael Zeng, and Xuedong Huang.
%    2018. Sdnet: Contextualized attention-based deep
%    network for conversational question answering.
%    CoRR, abs/1812.03593.
%
%    \bibitem{simpleqa}Yasuhito Ohsugi, Itsumi Saito, Kyosuke Nishida, 
%    Hisako Asano, Junji Tomita. 
%    A Simple but Effective Method to Incorporate 
%    Multi-turn Context with BERT for Conversational 
%    Machine Comprehension. In
%    Annual Meeting of the Association for Compu-
%    tational Linguistics (ACL) 2019.
%
%    \bibitem{Retrospective}Zhuosheng Zhang, Junjie Yang, Hai Zhao. 2020.
%    Retrospective Reader for Machine Reading Comprehension. In AAAI
%
%
%
%
%
%
%
%
%\end{thebibliography}
%%\end{multicols}

\section{总结与展望}
本文从机器阅读理解任务的定义出发,第二章概述了机器阅读理解任务以及介绍了不同任务下的数据集和相应的评估标准。
第三章神经机器阅读理解模型进行了分析与研究,主要涉及经典模型
的整体框架,其中详细分析了各个模型在交互层注意力机制的设计。此外还介绍了复杂任务下MRC模型的设计,同时也总结了
目前一些主流的预训练模型,分析了它们之间的差异,列举了一些在预训练模型的基础上改进
的MRC模型。通过各个模型的实验对比结果可以看到基于预训练的模型性能要显著的优于传统的仅仅基于
注意力机制的模型。第四章回顾了MRC领域的发展并且指出了目前MRC领域存在的问题。
%列举了一些目前MRC领域更加复杂的任务并且对每一种任务下相关的模型做了介绍。

机器阅读理解赋予了计算机阅读理解文本的能力,在搜索、对话、医疗以及教育领域都有着广阔的应用空间。

1)可解释性

2)目前机器阅读理解领域的数据集大多是通用领域方向的,而设计专业领域数据集也尤为重要,更重要的是
这些适用于通用领域数据集的模型未必在专业领域有一样的性能。

3)常识能力


4)目前机器阅读理解主要集中于非结构化的文本领域,而
还有许多其它结构,不同模态的数据如表格、视频、音频、图片等,
多模态阅读理解模型也是未来的发展方向之一。




\printbibliography[title={参考文献}]




\end{document}