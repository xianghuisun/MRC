\noindent \textbf{摘\quad 要}: 机器阅读理解
的目的是使得机器能够理解自然语言文本,
它是自然语言处理领域
十分重要的研究方向。
随着深度学习技术的进步以及大规模数据集的发布,在机器阅读理解方向
上的研究已经取得了很大的突破。近年来随着自然语言处理领域预训练模型的出现,再一次推动了机器阅读理解领域的发展。
本文主要从四个方面对自2015年以来机器阅读理解领域的发展做综述:介绍机器阅读理解的任务定义以及相关数据集;
分析机器阅读理解领域经典的基于注意力机制或推理结构的模型以及目前流行的预训练模型;
探讨更加复杂的机器阅读理解任务;
总结机器阅读理解领域目前存在的问题并且对未来的研究趋势做展望。\\
\heiti 关键词: \songti 机器阅读理解;自然语言处理;预训练模型;注意力机制;推理结构 \\
%\heiti 中图分类号: \songti ? \hspace{1cm} \heiti 文献表示码: \songti A \\
\textbf{文献标识码:} A  \qquad \textbf{中图分类号:} TP391
\begin{center}
    \textbf{\zihao{4} Overview of Studies on Neural Machine Reading Comprehension \\}

%     \zihao{5} Xianghui Sun \\
% \zihao{-5} id: 1971654 (NEU) 
\zihao{-4} SUN Xianghui \\
College of Computer Science and Engineering,Northeasten University,Shengyang 110169,China

\end{center}
\textbf{Abstract:} Machine Reading Comprehension aims to make machines comprehend the natural language documents, which 
is an important research direction
 in the field of natural language processing. With the development of deep learning technology and release of large scale datasets, the research on the field of machine reading comprehension has made great breakthroughs. 
 With the emergence of pre-trained model in natural language processing in recently years, it promotes the development of machine reading comprehension once again. This paper mainly make a survey from four aspects over the development of machine reading comprehension in recently years: to introduce the definition of 
  machine reading comprehension tasks and its corresponding datasets; to analyze classical model in the field of machine reading comprehension  
  which based on attention mechanism or reasoning structure as well as the currently 
  popular pre-trained model; to discuss more complicated machine reading comprehension tasks;
   to summarize the existing problem and look into the future research trend about machine reading comprehension. \\
\textbf{Key words:} machine reading comprehension; natural language processing; pre-trained model; attention mechanism; reasoning structure











